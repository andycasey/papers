%!TEX TS-program = pdflatex
\documentclass{emulateapj}

\usepackage{amssymb} % for resolution symbol
\usepackage{verbatim} % for comment blocks
\usepackage{pdflscape} % landscape tables

\shorttitle{The First High-Resolution Spectroscopic Analysis of the Orphan Stream}
\shortauthors{Casey et al}

\begin{document}

\title{Hunting the Parent of the Orphan Stream: The First High-Resolution Spectroscopic Analysis}

\author{Andrew R. Casey\altaffilmark{1}, Stefan C. Keller\altaffilmark{1}, Gary Da Costa\altaffilmark{1}, Frebel, Anna\altaffilmark{2}, and Elizabeth Maunder\altaffilmark{1}}
\altaffiltext{1}{Research School of Astronomy \& Astrophysics, Australian National University, Mount Stromlo Observatory, via Cotter Rd, Weston, ACT 2611, Australia; \email{acasey@mso.anu.edu.au}}
\altaffiltext{2}{Massachusetts Institute of Technology, Kavli Institute for Astrophysics and Space Research,
77 Massachusetts Avenue, Cambridge, MA 02139, USA}

\begin{abstract}


%We present the first high-resolution spectroscopic data for five Orphan Stream candidates, observed with the Magellan Inamori Kyocera Echelle (MIKE) spectrograph on the Magellan Clay telescope. 
%The candidates were selected from the low-resolution catalog of \citet{casey;et-al_2013a}: 3 high-probability members, 1 medium- and 1 low-probability stream candidate. Inferred stellar parameters and chemical abundances from our data indicate the low- and medium-probability target are metal-rich field stars. The remaining three high-probability targets are chemically distinct from our standard field stars.

%If the chemistry displayed in these stars is representative of the Orphan Stream and it's unidentified parent host, we have witnessed a unique chemical indicator that can be used to unambiguously trace the Orphan Stream throughout the Milky Way, and a `smoking gun' for the parent of the Orphan Stream.
\end{abstract}

\keywords{Galaxy: halo, structure --- Individual: Orphan Stream --- Stars: K-giants}

\section{Introduction}

% Galaxy formation in the context of LambdaCDM cosmology
%	-> Bottom-up formation
%	-> Merging
%	-> In situ/Accretion
%	-> Tidal tails

In $\Lambda$CDM cosmology, structure is formed hierarchically in a bottom-up fashion. Small star clusters conglomerate together to form larger systems like the Milky Way. Many stars within our galaxy have formed in-situ \citep{ELS}, but a significant fraction has been built up by the accretion of smaller systems, particularly in the halo \citep{searle;zinn_1978}. As these satellite systems infall towards the Milky Way potential they are disrupted by tidal forces, causing exterior stars to be strewn in forward and trailing directions as tidal tails or `stellar streams'. The evidence for this hierarchical formation and ongoing accretion in the Milky Way has increased significantly over the last decade \citep{ibata;et-al_1994;belokurov;et-al_2006b}.

% Level of accretion in the Milky Way
%	-> Significant fraction of the halo
%	-> With multi-band photometry across the sky the search for these substructures has become more efficient
%	-> Range of masses of infalling progenitors, enabling comparisons to CDM simulations, cosmology

%These discoveries have demonstrated that the Milky Way is the best laboratory to examine accretion in the context of galaxy formation and evolution. From the pencil-beam survey of \citet{starkenburg;et-al_2009}, they find a lower limit of 10\%  substructure for the halo. In contrast, \citet{bell;et-al_2008} suggest that the entire stellar halo has been formed by the accretion of smaller dwarf satellites. A close examination of high-resolution cosmological simulations suggests this number could exceed $>84$\%. More precise observational constraints await the results of large scale spectroscopic surveys. Irrespective of the overall fraction, it is generally accepted that a considerable amount of the halo has been built up by the disruption of smaller systems.

% Deason
% Sharma & Johnson
% Schlaufman


% The Orphan Stream and its properties
%	-> Detected in the SDSS data set
%	-> Extent on sky
%	-> Stream width
%	-> Implications from the stream width and extent: dark matter dominated, old, metal-poor, velocity dispersion, progenitor mass.

Amongst the known substructures in the halo, the Orphan Stream is particularly interesting. \citet{grillmair_2006} and \citet{Belokurov;et-al_2006} independently discovered the stream, spanning over $60^\circ$ in the sky from Ursa Major in the north to near Sextans in the south. The stream has properties that are unique from other halo substructures. It has an extremely low surface brightness ranging from 32-40\,mag arcsec$^{-2}$, and a full width half-mass of $\sim$700\,pc at a distance of $\sim$20\,kpc, which is approximately $\approx2^\circ$ when projected on the sky \citep{belokurov;et-al_2007,grillmair_2006}. The cross-section and luminosity of the stream are directly related to the mass and velocity dispersion of the parent satellite \citep{johnston_1998}. The Orphan Stream width is significantly broader than every known globular cluster tidal tail \citep{odenkirchen;et-al_2003,grillmair_johnson_2006,grillmair_dionatos_2006a,grillmair_dionatos_2006b} and larger than the tidal diameter of all known globular clusters \citep{harris}. As \citet{grillmair_2006} notes, if the cross-section of the stream is circular, then in a simple logarithmic potential with $v_c = 220$\,km s$^{-1}$ -- a reasonable first-order approximation -- the expected random velocities of stars would be required to be $\Delta{E} > 20$\,km s$^{-1}$ in order to produce the stream width. Such a velocity dispersion is significantly larger than the expected random motions of stars that have been weakly stripped from a globular cluster, implying that the Orphan Stream's progenitor mass must be much larger than a classical globular cluster. Photometry indicates the stream is metal-poor, implying that negligible star formation has occurred since infall began several gigayears ago. For a stream of this length to remain structurally coherent over such a long timescale, the progenitor is also likely to be dark-matter dominated. In their discovery papers, \citet{grillmair_2006} and \citet{belokurov;et-al_2007} concluded that the likely parent of the Orphan Stream is a low-luminosity dwarf spheroidal galaxy.

% Observational efforts
%	-> Belokurov 2007
%	-> Newberg's work on BHB stars
%	-> Distances, velocities, strong distance and velocity gradient
%	-> Casey 2013a metallicities, velocities
 
\citet{newberg;et-al_2010} were able to map the distance and velocity of the stream across the length of the SDSS catalogue using blue horizontal branch (BHB) and F-turnoff stars. They found the stream distance to vary between 19-47\,kpc. This extended further than the 20-32\,kpc measurements made by \citet{belokurov;et-al_2007}. \citet{newberg;et-al_2010} note an increase in the density of the Orphan Stream near the celestial equator $(l, b) = (253\,^\circ, 49\,^\circ)$, proposing the progenitor may be close to this position. The authors attempted to extend their trace of the stream using the southern SuperCOSMOS data set \citep{supercosmos}, but to no avail; the stream's surface brightness is too low. It is still unclear whether the stream extends deep into the southern sky. \citet{newberg;et-al_2010} note an increase in surface brightness near the celestial equator, but given the stream is closest at the equator \citep{belokurov;et-al_2007}, an increase in stream density may be expected given a constant absolute magnitude. Nevertheless, with SDSS photometry and radial velocities from the SEGUE catalog, \citet{newberg;et-al_2010} were able to derive a prograde orbit with an eccentricity, apogalacticon and perigalacticon of $e = 0.7$, 90\,kpc and 16.4\,kpc respectively. If this orbit is accurate, the ability to trace a stellar stream out to 90\,kpc would make a powerful probe for measuring the galactic potential. From their simulations, \citet{newberg;et-al_2010} find a halo and disk mass of $M(R < 60\,{\rm kpc}) = 2.6 \times 10^{11} M_\odot$, $\sim$60\% lower than that found by \citet{xue;et-al_2008} and \citet{koposov;et-al_2008}, and slightly lower than the virial mass of $7 \times 10^{11} M_\odot$ found by \citet{sales;et-al_2008}.
 
Metallicities derived from SEGUE low-resolution spectra confirm photometric estimates indicating that the stream is metal-poor. A mean metallicity of [Fe/H]$ = -2.1 \pm 0.1$\,dex is found from BHB stars, with a range extending from $-$1.3 to $\sim-$3\,dex \citep{newberg;et-al_2010}. If F-turnoff stars from that sample are included, the metallicity distribution function extends more metal-rich from $\sim-$3 to $-$0.5\,dex. Since BHB stars only form from low metallicity populations, it is not unexpected for BHB stars to be more metal-poor than F-type giants. 
% Sesar work

The situation is further complicated by halo interlopers and small number statistics, so the full shape of the metallicity distribution function (MDF) is unknown. To this end, \citet{casey;et-al_2013a} observed low-resolution spectroscopy for hundreds of stars towards the stream at the celestial equator. The authors targeted the less numerous K-giants \citep[a mere 1.3 red giant branch stars expected per square degree]{sales;et-al_2008,morrison_1993} and found a very weak detection of the stream from kinematics alone. Using wide selections in velocity, distance, proper motions, metallicities, and surface gravity, they identified highly likely Orphan Stream giants. The velocity dispersion of their candidates is within the observational errors, suggesting the stream is kinematically cold along the line-of-sight. Like \citet{newberg;et-al_2010} (and now additionally confirmed by \citet{sesar;et-al_2013}), they also found an extended range in metallicities of their nine most probable candidates: two stars below $\leq-2.70$ and two stars near $-1.17$\,dex all of which are consistent with stream membership. The mean metallicity of their sample was [Fe/H]$ = -1.63 \pm 0.19$\,dex, with a wide dispersion of $\sigma({\rm [Fe/H]}) = 0.56$\,dex. It appears the Orphan Stream may have an extremely wide range in metallicity, consistent with the stochastic chemical enrichment typically observed in dwarf spheroidal (dSph) galaxies \citep{mateo_1998}.

% Possible associations
%	-> There are a number of globular clusters associated along the great circle path.
%	-> Belokurov speculated some association between UM II and Complex A, suggesting the progenitor is a dSph and the clusters have been flung off
%	-> Complex A models by Jin & Lynden-Bell, but they incorrectly predict velocities and under-estimate distances by a factor of three.
%	-> Ursa Major II simulations do not match
%	-> Rule out all known satellites except for Segue 1

As the name suggests, the Orphan Stream's parent satellite has yet to be found. In an effort to identify a progenitor, a number of systems have been identified along the great circle path of the stream as being plausibly associated with the Orphan Stream. These include the linear Complex A H\,\textsc{i} clouds, as well as the globular clusters Palomar 1, Apr 2, Terzan 7 and Ruprecht 106. The low-luminosity dwarf satellites Segue 1 and Ursa Major II also lie along the great circle, although Segue 1 was considered an extended globular cluster until recently \citep{norris;et-al_2010,simon;et-al_2011}. It is worth noting that streams do not often reside along a great circle from the observer's perspective, nor are they expected to do so. \citet{belokurov;et-al_2007} first noted a possible association between the Orphan Stream, Ursa Major II and Complex A. To that end \citet{Jin;Lynden-Bell_2007} and \citet{fellhaur;et-al_2007} explored the possible association between the Orphan Stream with Complex A and Ursa Major II, respectively. In the best-fitting Complex A model, the predicted heliocentric velocities did not match those found by \citet{belokurov;et-al_2007}, or later observations by \citet{newberg;et-al_2010}. The expected distances were also under-estimated by a factor of $\sim{}3$ compared to observations, making the association with Complex A tenuous at best. In the Ursa Major II scenario, the stream's on-sky position was required to exactly overlap with a previous wrap, a somewhat contrived and unlikely scenario. \citet{newberg;et-al_2010} found that the Ursa Major II-Orphan Stream connection was also not compelling as the stream kinematics were not consistent with the Ursa Major II model.

Simulations involving the Complex A and Ursa Major II associations introduced an a priori assumption that the object (e.g. Ursa Major II or Complex A) was related to the Orphan Stream, and consequently found an orbit to match. In contrast, \citet{sales;et-al_2008} approached the problem by fitting an orbit to a single wrap of the data, without assuming a parent satellite a priori. Their $N$-body simulations were inconsistent with either a Complex A or Ursa Major II association with the Orphan Stream. Instead, the authors favour a progenitor with a luminosity $L \sim 2.3 \times 10^4 L_\odot$ or an absolute magnitude $M_r \sim -6.4$, consistent with the observation by \citet{belokurov;et-al_2007} of $M_r \sim -6.7$. Simulations by \citet{sales;et-al_2008} suggest the progenitor may be similar to the present day `classical' Milky Way dwarfs like Carina, Draco, Leo II or Sculptor, but would be very close to fully disrupted over infall, which they suggest has occurred over the last 5.3 Gyr. Time of infall is a critical inference. Longer timescales produce streams that are too wide and diffuse, whereas shorter timescales do not reproduce the $\sim{}$60\,$^\circ$ stream length. The degeneracies between these simulation parameters are important to note, but in any case there are robust conclusions that can be drawn irrespective of the degeneracies. For example, \citet{sales;et-al_2008} note that a globular cluster has a central density too high ($\sim10^{12} M_\odot$\,kpc$^{-3}$) to be fully disrupted along their Orphan Stream orbit within a Hubble time. Given this constraint and the lower limit on luminosity ($L > 2\times10^5 L_\odot$), a globular cluster progenitor seems unlikely from their models.

In addition to the work by \citet{sales;et-al_2008}, $N$-body simulations by \citet{newberg;et-al_2010} exclude all known halo globular clusters as possible progenitors. They conclude with the postulation of two possible scenarios: the progenitor is an undiscovered satellite located between $(l, b) = (250^\circ, 50^\circ)$ and $(270^\circ, 40^\circ)$, or Segue 1 is the parent system. 

Segue 1 is an extremely dark-matter dominated system. In fact, kinematics suggest it is so dark-matter dominated that it has become a prime focus for the detection of direct dark-matter annihilation experiments \citep{dm_dominated}.

Segue 1 itself has an extremely wide range in metallicities, varying from [Fe/H] = X to Y dex. 


% Segue 1 association?
%	-> Segue 1 is a dark-matter dominated system
%	-> Velocities and positions
%	-> Does it match the expected L, M values from constraints on models
%	-> Extremely wide metallicity range.



% NGC 2419

Given the extended apogalaction of 90\,kpc in the stream orbit found by \citet{newberg;et-al_2010}, \citet{bruns;kroupa_2011} reasoned the stream may be the tidal tail of the massive globular cluster NGC 2419. This system is the most distant (90 kpc) -- yet one of the most massive (R) and luminous ($M$) -- globular clusters known. It is unlike any other globular cluster in the Milky Way. It's shear distance makes spectroscopic observations challenging, but work by \citet{cohen;et-al} have confirmed photometric observations \citet{photometry_for_ngc2419} that the system is metal-poor ([Fe/H] $= -2?$), and identified a remarkable anti-correlation between Mg and K. The level of magnesium depletion ([Mg/Fe] $= -0.40$\,dex??) is not seen anywhere else in the galaxy, neither has the enhanced potassium enrichment ([K/Fe] = $1.5$\,dex??) at the converse end of this peculiar Mg-K anti-correlation. This is also a difficult pattern to explain with from a nucleosynthesis perspective (A. Karakas, in preparation). If NGC 2419 is the parent of the Orphan Stream then this unprecedented chemical signature ought to exist in disrupted stream members as an example of chemical tagging \citep[e.g. see][]{freeman;bland-hawthorn_2002,de_silva;et-al_2007,wylie-de-boer;et-al_2010,majewski;et-al_2012}.

% For nice formatting, the LaTeX table is here:
\begin{deluxetable*}{lccccccccccc}
\tablecolumns{2}
\tabletypesize{\scriptsize}
\tablecaption{Observations\label{tab:observations}}
\tablehead{
	\colhead{Object} &
	\colhead{$\alpha$} &
	\colhead{$\delta$} &
	\colhead{$g$} &
	\colhead{UT Date} &
	\colhead{UT Time} &
	\colhead{Airmass} &
	\colhead{Exp. Time} &
	\colhead{S/N\tablenotemark{a}} &
	\colhead{$V_{\mbox{hel}}$} &
	\colhead{$V_{\mbox{err}}$} \\
 & (J2000) & (J2000) & (mags) & & & (") & (secs) & (px$^{-1}$) & (km s$^{-1}$) & (km s$^{-1}$)
}
\startdata
HD 41667		& 06:05:03.7	& $-$32:59:36.8	& 		& 2011-03-13	& 23:40:52	& 1.005	& 90			& 272	& 297.8 	& 1.7 \\
HD 44007		& 06:18:48.6	& $-$14:50:44.2	& 		& 2011-03-13	& 23:52:18	& 1.033	& 30 			& 239	& 163.4 	& 1.3 \\
HD 76932		& 08:58:44.2	& $-$16:07:54.2	& 		& 2011-03-14	& 00:16:47	& 1.158	& 23 			& 289	& 119.2 	& 1.2 \\
HD 122563		& 14:02:31.8	& $+$09:41:09.9	\\
HD 136316		& 15:22:17.2	& $-$53:14:13.9	& 		& 2011-03-14	& 09:37:26	& 1.118	& 3060 		& 335	& -38.2	& 1.1 \\
HD 141531		& 15:49:16.9	& $+$09:36:42.5	& 		& 2011-03-14	& 09:52:00	& 1.309	& 9180 		& 280	& 2.6 	& 1.0 \\
HD 142948		& 16:00:01.6	& $-$53:51:04.1	& 		& 2011-03-14	& 09:45:12	& 1.107	& 3060 		& 271	& 30.3 	& 0.9 \\
OSS-3 (1)		& 10:46:50.6	& $-$00:13:17.9	& 17.33 	& 2011-03-14	& 01:51:07	& 1.363	&4$\times$2500	& 48	& 217.9	& 0.9 \\ 
OSS-6 (2)		& 10:47:17.8	& $+$00:25:06.9	& 16.09 	& 2011-03-14	& 00:25:37	& 1.995	&3$\times$1600	& 59	& 221.2	& 0.4 \\ 
OSS-8 (3)		& 10:47:30.3	& $-$00:01:22.6	& 17.25	& 2011-03-14	& 04:44:04	& 1.160	&5$\times$1900	& 49	& 225.9	& 0.5 \\ 
OSS-14 (4)		& 10:49:08.3	& $+$00:01:59.3	& 16.27 	& 2011-03-15	& 00:32:17	& 1.881	&4$\times$1400	& 48 	& 225.1	& 0.3 \\ 
OSS-18 (5)		& 10:50:33.7	& $+$00:12:18.3	& 17.82 	& 2011-03-15	& 02:12:46	& 1.295	&4$\times$2100	& 31 	& 247.8	& 0.6
\enddata
\tablenotetext{a}{S/N measured at 600\,nm for each target.}
\end{deluxetable*}



% What we report here.
In this study we present an analysis of high-resolution spectroscopic observations for five Orphan Stream candidates and seven well-studied field stars with similar stellar parameters for comparison. The observations and data reduction are outlined in Section \ref{sec:observations}. In Section \ref{sec:analysis} we describe the details of our curve-of-growth analysis and spectral synthesis approach to infer stellar parameters and chemical abundances. We discuss the results of our analysis in Section \ref{sec:discussion}, including the implications for association between the Orphan Stream and the two currently alleged stream progenitors: Segue 1 and NGC 2419. Finally, we conclude in Section \ref{sec:conclusions} with a summary of our findings.



\section{Observations and Data Reduction}
\label{sec:observations}

%HD 140283	& 15:43:02.3	& $-$10:56:03.0	&		& 2010-08-06	& 23:02:19	& 1.056	& \nodata		& \nodata			& \nodata \\
%l = 250.21, b = 49.46, Vrad = 222.5 +/- 0.9 km/s, Orphan_Lambda = 19.11, Orphan_Beta = -0.48, Vgsr = 83.96 +/- 0.9 km/s
%l = 249.63, b = 49.99, Vrad = 225.7 +/- 0.4 km/s, Orphan_Lambda = 18.54, Orphan_Beta = -0.16, Vgsr = 89.23 +/- 0.4 km/s
%l = 250.17, b = 49.71, Vrad = 230.9 +/- 0.9 km/s, Orphan_Lambda = 18.98, Orphan_Beta = -0.27, Vgsr = 93.15 +/- 0.9 km/s
%l = 250.56, b = 50.04, Vrad = 229.7 +/- 0.3 km/s, Orphan_Lambda = 19.06, Orphan_Beta = +0.14, Vgsr = 92.65 +/- 0.3 km/s
%l = 250.76, b = 50.41, Vrad = 252.6 +/- 0.6 km/s, Orphan_Lambda = 19.02, Orphan_Beta = +0.53


Five Orphan Stream candidates and six well-studied standard stars were observed in March 2011 using the 1\arcsec slit. An additional metal-poor standard star, HD 122563, was observed in similar conditions in X 20YY. This slit configuration provides a spectral resolution of $\mathcal{R} = 22,000$ in the blue arm and $\mathcal{R} = 28,000$ in the red arm. According to \citet{casey;et-al_2013a}, tjree of the selected stream candidates have high a probability of membership, one was classified with medium probability, and another with a low probability of membership. Initially we planned to observe many more high-priority targets. However, after our last exposure of OSS 18, inclement weather forced us to relinquish the remainder of our observing time, losing almost half of the run. The details of our observations are tabulated in Table \ref{tab:observations}. A minimum of 10 exposures of each calibration type (biases, flat fields, and diffuse flats) were observed in the afternoon of each day, with additional flat-field and Th-Ar arc lamp exposures performed throughout the night to ensure an accurate wavelength calibration.


% observed using the Magellan Inamori Kyocera Echelle (MIKE) spectrograph on the Magellan Clay telescope. These targets are candidate K-giants identified to be likely stream members on the basis of position, velocity, metallicity and surface gravity from the low-resolution spectroscopic study of \citet{casey;et-al_2013a}.

The data were reduced using the CarPy pipeline written by D. Kelson\footnote{D. Kelson website}. Every reduced echelle aperture was carefully normalised using cubic splines with defined knot spacing. Extracted apertures were stacked together and weighted by their flux counts to provide a continuous normalised spectrum for each observation.


\section{Analysis}
\label{sec:analysis}
Each normalised, stitched spectrum was cross-correlated against a synthetic template to measure radial velocities. This was performed using a Python implementation of the \citet{Tonry;Davis_1978} method. The wavelength region employed was from $845 \leq \lambda \leq 870$\,nm, and a K2V metal-poor giant synthetic spectrum was used as the rest template. Heliocentric corrections have been applied to the our radial velocity measurements, and the results are shown in Table \ref{tab:observations}.

% line by line velocities? velocities consistent with Casey 2013a?

Equivalent widths (EWs) for all atomic transitions listed in Table \ref{tab:atomic-data} were measured using the automatic profile fitting algorithm described in \citet{casey;et-al_2013b}. Although this technique is accurate and robust against blended lines and strong changes in local continuum \citep{frebel;et-al_2013a,casey;et-al_2013b,keller;et-al_2013,frebel;et-al_2013c}, every fitted profile was visually inspected for quality. Spurious or false-positive measurements were removed, and of order $\sim$5 interactive EW measurements were necessary for each spectrum.

Given the range in metallicities for the Orphan Stream candidates (see Section \ref{sec:analysis-metallicity}), this meant that only fourteen Fe\,\textsc{i} and four Fe\,\textsc{ii} transitions in our line list were measurable for our most metal-poor candidate, OSS 4. With so few lines available, minute changes in stellar parameters will result in significant variations to the abundance trends during excitation and ionization equilibria, and a poor overall fit to the model. At this point we opted to supplement our line list with additional transitions from \citet{roederer;et-al_2010}. Each added transition was inspected in our most metal-rich candidate (OSS 1) to ensure that it was not blended with other features. The minimum number of usable Fe\,\textsc{i} and Fe\,\textsc{ii} transitions for any star increased to X and Y, respectively.

\begin{deluxetable*}{crcrccccccccccc}[t!]
\tabletypesize{\scriptsize}
\tablecolumns{2}\tablecaption{Equivalent Widths\label{tab:equivalent-widths}}
\tablehead{
    \colhead{Wavelength} &
    \colhead{Species} &
    \colhead{$\chi$} &
    \colhead{$\log{gf}$} &
    \colhead{HD 41667} &
    \colhead{HD 44007} &
    \colhead{HD 76932} &
    \colhead{HD 122563} &
    \colhead{HD 136316} &
    \colhead{HD 141531} &
    \colhead{HD 142948} &
    \colhead{(Cont..)} \\
 ({\AA}) & & (eV) & & (m{\AA}) & (m{\AA}) & (m{\AA}) & (m{\AA}) & (m{\AA}) & (m{\AA}) & (m{\AA}) &
}
\startdata
 6300.300 &       8.0 &      0.00 &    $-$9.717 &   \nodata &     14.79 &   \nodata &      6.77 &   \nodata &     38.16 &     35.37 &   \nodata \\
 6363.780 &       8.0 &      0.02 &   $-$10.185 &     11.35 &   \nodata &   \nodata &   \nodata &     11.58 &     14.74 &     12.70 &   \nodata \\
 5688.190 &      11.0 &      2.11 &    $-$0.420 &     71.85 &     30.87 &     57.44 &   \nodata &     29.30 &     38.07 &    117.41 &     55.17 \\
 6154.230 &      11.0 &      2.10 &    $-$1.530 &     10.66 &   \nodata &      7.75 &   \nodata &   \nodata &      5.23 &     28.38 &   \nodata \\
 6160.750 &      11.0 &      2.10 &    $-$1.230 &     16.55 &   \nodata &     13.38 &   \nodata &      4.58 &      6.98 &     43.40 &     17.39 \\
 3829.355 &      12.0 &      2.71 &    $-$0.208 &   \nodata &   \nodata &   \nodata &    190.12 &   \nodata &   \nodata &   \nodata &   \nodata \\
 3832.304 &      12.0 &      2.71 &     0.270 &   \nodata &   \nodata &   \nodata &    226.99 &   \nodata &   \nodata &   \nodata &   \nodata
\enddata
\tablenotetext{}{Table \ref{tab:equivalent-widths} is published in its entirety in the electronic edition. A portion is shown here for guidance regarding its form and content.}
\end{deluxetable*}

% stellar parameter determination..
\subsection{Stellar Parameters}
We have employed the 1D plane-parallel model atmospheres of \citet{castelli;kurucz_2004} to infer stellar parameters from the EWs of atomic absorption profiles. These $\alpha$-enhanced models assume that absorption lines form under local thermal equilibrium and ignore convective overshoot as well as center-to-limb variations. We have interpolated within a grid of these model atmospheres following the prescription in \citet{casey;et-al_2013b}.

\subsubsection{Effective Temperature, $T_{\rm eff}$}
Effective temperatures for all stars have been found by excitation balance of neutral Fe lines. Transitions with reduced equivalent widths (REW), $\log(\textrm{EW}/\lambda) > -4.5$, were excluded to avoid using lines near the flat region of the curve-of-growth. Linear fits to the data with gradients less than $|10^{-3}|$\,dex eV$^{-1}$ were considered to be converged, which equated to a minimum temperature step size of $\sim$5\,K. Since all EW measurements had been visually inspected, we generally identified almost no outlier measurements during the excitation or ionization balance analyses. The most number of Fe\,\textsc{i} outliers ($>3\sigma$) removed while determining the effective temperature was three.

% Photometric temperatures for standard stars?

% Effective temperatures for standard stars
The effective temperatures for our standard stars is in excellent agreement with those presented in the literature. The largest discrepancy occurs HD 122563, a cool metal-poor giant. For metal-poor giants, temperature found through excitation balance are known to produce systematically cooler temperatures than those deduced from the infra-red flux method. Similarly, the resultant surface gravities are also lower \citep[see][for a discussion]{frebel;et-al_2013a}. We have chosen to remain consistent with the excitation balance method and accept the systematically cooler temperature of 4360\,K. In contrast, for HD 142948 we find a 307\,K hotter temperature than that found by \citet{gratton;et-al_2000}. The reason for this discrepancy is unclear. Excluding these two stars we find the mean offset in temperatures is 24\,K.

%On the basis of spectral typing, we confirm the high probability stream candidates OSS 6 and 14 are indeed K-giants, as indicated by \citet{casey;et-al_2013a}. The final high probability candidate, OSS 8, is slightly hotter than expected: a G?? giant. The remaining lower probability candidates have an effective temperature $\sim$5200, indicating they are F-turnoff giants.

\subsubsection{Microturbulence, $\xi$}
Microturbulence is necessary in 1D model atmospheres to represent large scale, 3D turbulent motions. The correct microturbulence will ensure that weak and strong transitions yield the same abundance. We have solved for the microturbulence by demanding a zero-trend in REW and line abundance for all Fe\,\textsc{i} lines. This process is iteratively conducted while solving for effective temperature to provide a full-model fit to the data. The resultant trend line slope between REW and abundance is typically $<|0.001|$\,dex, although in some cases this could not be achieved without increasing the microturbulence precision to three decimals. The largest slope in any star is $-0.004$\,dex. Plots of excitation potential, reduced equivalent width and abundance for the program stars are shown in Figure \ref{fig:ionization-excitation-balance}.

% Figure for ionization and excitation balance for standard stars
% Include slope, uncertainty, correlation coefficient and probability.


\subsubsection{Surface Gravity, $\log{g}$}
We have inferred surface gravities through the ionization balance of neutral and singly ionized Fe lines. This process was iteratively conducted while solving for other stellar parameters. All standard and program stars are RGB stars. By comparison of surface gravities in published surface gravities for our standard stars, [DOES IT??] it appears our surface gravities are $\sim{}-$0.20\,dex lower than that found by other authors \citep{gratton;et-al_2000,fulbright_2000,nissen;schuster_2000;gratton_sneden_1991}. The source of this discrepancy is unclear, yet the difference remains well within the 1$\sigma$ uncertainties for either study.


\begin{deluxetable*}{lcccccccccccccl}[h!]
\tablecolumns{2}
\tabletypesize{\scriptsize}
\tablecaption{Stellar Parameters\label{tab:stellar-parameters}}
\tablehead{
	\colhead{Object\tablenotemark{a}} &
	\colhead{$T_{\rm eff}$} &
	\colhead{$\log{g}$} &
	\colhead{$\xi$} &
	\colhead{[Fe/H]} & 
	\colhead{$T_{\rm eff}$} &
	\colhead{$\log{g}$} &
	\colhead{$\xi$} &
	\colhead{[Fe/H]} &
	\colhead{Literature Source} \\
	& (K) & (dex) & (km\,s$^{-1}$) & (dex) & (K) & (dex) & (km\,s$^{-1}$) & (dex) 
}
\startdata
\\
\multicolumn{10}{c}{Standard Stars} \\
\hline
HD 41667		& 4643	& 1.54	& 1.81	& $-$1.18 & 4605 & 1.88 & 1.44 & $-$1.16 & \citet{gratton;et-al_2000} 		\\ % +38
HD 44007		& 4820	& 1.66	& 1.70	& $-$1.69 & 4850 & 2.00 & 2.20 & $-$1.71 & \citet{fulbright_2000} 		\\ % -30
HD 76932		& 5835	& 3.93	& 1.42	& $-$0.95 & 5849 & 4.11 & \nodata & $-$0.88 & \citet{nissen;et-al_2000} 	\\ % -14
HD 122563		& 4358	& 0.14	& 2.77	& $-$2.90 & \\
HD 136316		& 4347	& 0.44	& 2.15	& $-$1.93 & 4414 & 0.94 & 1.70 & $-$1.90 & \citet{gratton_sneden_1991} 	\\ % -41
HD 141531		& 4373	& 0.52	& 2.05	& $-$1.65 & 4280 & 0.70 & 1.60 & $-$1.68 & \citet{shetrone_1996} 		\\ % +93
HD 142948		& 5050	& 2.39	& 1.83	& $-$0.64 & 4713 & 2.17 & 1.38 & $-$0.77 & \citet{gratton;et-al_2000} 		\\ % +337

\\
\multicolumn{10}{c}{Orphan Stream Candidates} \\
\hline
OSS 3 (L)		& 5225	& 3.16	& 1.10	& $-$0.86 & \nodata	& \nodata	& \nodata	& $-$1.31	& \citet{casey;et-al_2013a}\\
OSS 6 (H)		& 4554	& 0.70	& 2.00	& $-$1.75 & \nodata	& \nodata	& \nodata	& $-$1.84	& \citet{casey;et-al_2013a}\\
OSS 8 (H)		& 4880	& 1.71	& 1.86	& $-$1.62 & \nodata	& \nodata	& \nodata	& $-$1.62	& \citet{casey;et-al_2013a}\\
OSS 14 (H)		& 4675	& 1.00	& 2.53	& $-$2.66 & \nodata	& \nodata	& \nodata	& $-$2.70	& \citet{casey;et-al_2013a}\\
OSS 18 (M)		& 5205	& 2.91	& 1.88	& $-$0.62 & \nodata	& \nodata	& \nodata	& $-$0.90	& \citet{casey;et-al_2013a}
\enddata
\tablenotetext{a}{Probability of membership (Low, Medium, High) listed for Orphan Stream candidates as defined by \citet{casey;et-al_2013a}.}
\end{deluxetable*}
% Mean temperatures are +68 K hotter than literature (not HD122563)
% Excluding HD 142948 (and HD122563) gives a mean of +9 K hotter than literature.

% Metallicities are +0.01 dex higher than literature for all standards (not HD122563) 

% Surface gravities are a mean -0.22 dex lower than literature (not HD122563)



% isochrone fits?

\subsubsection{Metallicity, $[M/H]$}
\label{sec:analysis-metallicity}

The final stage of the stellar parameter determination is to derive the total metallicity, as it has the least dependence on stellar parameters. For these analyses we adopt the mean [Fe\,\textsc{i}/H] abundance as the overall metallicity [M/Fe]. A difference of ${|\textrm{[M/H]} - \langle\textrm{[Fe\,\textsc{i}/H]}\rangle| \leq 0.01}$\,dex was considered an acceptable fit to the data. 

Excluding HD 122563, the mean difference in metallicity for all of our standard stars when compared to literature is a negligible $-$0.02\,dex with a standard deviation of 0.05\,dex. The largest difference is 0.08\,dex for HD 41667 \citep{gratton;et-al_2000}, still easily within either study's 1$\sigma$ uncertainties. Therefore, we are reasonably confident with our stellar parameters and metallicities inferred from our high-resolution spectroscopic data.

The difference in metallicities between the values we derive from high-resolution spectroscopy and those found by \citet{casey;et-al_2013a} from low-resolution spectroscopy are noticeable. For the high and medium probability targets (OSS 6, 8, 14 and 18) the agreement is reasonable: the values in this study are up to $+0.15$\,dex more metal-rich. We note that the uncertainties adopted in \citet{casey;et-al_2013a} are of the order $\pm0.3$\,dex. The largest difference between this study and that of \citet{casey;et-al_2013a} is in the lowest probability target (OSS 3), where we find a metallicity that is $+$0.45\,dex higher. For our high-resolution spectra, it is somewhat expected that the high and medium targets would have metallicities more agreeable with the low-resolution measurements. This is because the low-resolution metallicities of \citet{casey;et-al_2013a} are dependent on the assumption that these stars are at the distance of the Orphan Stream, approximately $\sim$20\,kpc away. Stars that showed any inconsistency with this assumption were marked with a lower probability of membership.



\subsection{Stellar Parameter Uncertainties}



\begin{deluxetable}{lcccccccccccccl}[h!]
\tablecolumns{1}
\tabletypesize{\scriptsize}
\tablecaption{Uncorrelated Uncertainties in Stellar Parameters\label{tab:stellar-parameter-uncertainties}}
\tablehead{
	\colhead{Object} &
	\colhead{$\sigma(T_{\rm eff})$} &
	\colhead{$\sigma(\log{g})$} &
	\colhead{$\sigma(\xi)$} \\
	& (K) & (dex) & (km\,s$^{-1}$)
}
\startdata
HD 41667	&	47 &	0.08 &	0.05 \\
HD 44007	&	51 &	0.08 &	0.17 \\
HD 76932	&	61 &	0.10 &	0.14 \\
HD 122563	&	15 &	0.01 &	0.07 \\
HD 136316	&	34 &	0.02 &	0.05 \\
HD 141531	&	40 &	0.03 &	0.05 \\
HD 142948	&	62 &	0.12 & 0.09 \\
OSS 3 (L)	&	73 &	0.08 &	0.11 \\
OSS 6 (H)	&	39 & 	0.04 &	0.15 \\
OSS 8 (H)	&	74 & 	0.17 &	0.23 \\
OSS 14 (H)	&	51 & 	0.05 &	0.13 \\
OSS 18 (M)	&	135& 	0.15 &	0.28
\enddata
\end{deluxetable}

\subsection{Chemical Abundances}
Chemical abundances are described following standard nomenclature, and comparisons are made against the Sun using the solar chemical composition described in \citet{asplund;et-al_2009}.

\subsubsection{Carbon}
Carbon has been measured from synthesizing spectra around the 431.3\,nm and 432.3\,nm CH band heads. The abundances inferred from these two syntheses were within 0.05\,dex. Nevertheless, have adopted a conservative uncertainty of $\pm$0.15\,dex in these measurements. The mid-point of these two measurements is listed in Table \ref{tab:chemical-abundances}. A portion of the synthetic and observed spectrum for HD XXXXXX and OSS 14 around the 431\,nm region is shown in Figure \ref{fig:carbon-synth}. We note that only the 432.3\,nm region was synthesized for OSS 18, as significant line blanketing was present in the bluer band-head and we deemed the 432.3\,nm region to yield a more precise determination of C-abundance than the 431.3\,nm portion. The [C/Fe] and [Fe/H] abundances for all standard and program stars is shown in Figure \ref{fig:carbon-fe}.

\subsubsection{Sodium, Aluminium and Potassium}
Sodium, aluminium and potassium are odd-Z elements that are primarily produced in massive stars through neutron capture reactions. Although our line list includes three clean, unblended Na lines, not all were detectable in the Orphan Stream candidates. Generally only one line was available for the program stars, and in most cases all were measurable in the standard stars. When multiple lines are present, the standard deviation is quite low: a maximum of $\sigma($[Na/Fe])$ = 0.06$\,dex is observed for {HD 41667}.

Since the Orphan stream stars are reasonably metal-poor, generally only one Al line was available for measurement. In these circumstances we adopt an uncertainty of $\pm0.1$\,dex. In some cases -- particularly for the metal-poor stars -- no aluminium lines were detectable and an upper limit was derived from synthesis of the strongest Al line. In contrast, at least one K line was detected in every standard and program star. These lines at 766 and 769\,nm are very strong, but fall directly between a strong telluric band head. Often both lines were detected, but one was dominated by the Earth's atmospheric absorption. In these cases we simply rejected the contaminated line and used the single, unaffected K line.  


\subsubsection{$\alpha$-elements (O, Mg, Ti, Si, Ca)}

Generally, the Orphan Stream candidates have lower $\alpha$-element abundances compared to iron than their halo counterparts. The situation is a little ambiguous for [O/Fe] compared to other $\alpha$-elements. We have employed the forbidden lines at 630 and 636\,nm to measure Oxygen abundances in our stars. Given the weakness of these lines, Oxygen was immeasurable in all of the Orphan stream candidates. In place of abundances, upper limits have been determined from spectral synthesis of the region. Our synthesis line list includes the Ni\,\textsc{i} feature hidden within the 630\,nm absorption profile \citep{allende-prieto;et-al_2001}. In the case of OSS 18, the forbidden oxygen region was sufficiently dominated by telluric absorption such that we deemed a robust upper limit to be indeterminable.

The [Mg/Fe] abundance ratios for the Orphan Stream stars are noticeably lower than any of the other $\alpha$-elements. In fact, the mean [$\alpha$/Fe] abundances are largely driven by [Mg/Fe]. Figure \ref{fig:alpha-fe} shows the $\alpha$-element abundances with respect to iron for all stream and halo targets. Ti \textsc{i} and Ti \textsc{ii} behave similarly to one another in each star, with the neutral species having slightly lower abundances. [Ca/Fe] is marginally lower in the stream candidates than their halo counterparts. 

\begin{figure}[h]
	\includegraphics[width=\columnwidth]{./alpha-Fe.pdf}
	\caption{$\alpha$-element abundance ratios with respect to Fe for standard halo (red) stars and Orphan Stream candidates (blue).}
	\label{fig:alpha-fe}
\end{figure}

\subsubsection{Fe-peak elements (V-Zn)}

The Fe-peak elements (V-Zn, $Z = 23$ to 30) are produced in supernova and generally trend in lockstep with Fe abundance. Most of our Fe-peak abundances show a near-zero trend with [Fe/H] (Figure \ref{fig:fe-peak}) with the exception of Cu. Only upper limits are available for our most metal-poor standard and Orphan stream star, but their limits extend the trend of decreasing [Cu/Fe] with [Fe/H]. Chromium and Manganese are slightly depleted with respect to [Fe/H] and the other Fe-peak elements, and they also show a subtle positive trend with [Fe/H].

The line-to-line scatter in each abundance ratio is quite small, on the order of 0.05\,dex. For [Co/Fe] we find a slightly larger scatter, up to $\sim$0.25\,dex in OSS 14. The fitted absorption profiles were of good quality and there was no reason to discard one measurement over another, so all abundances were kept and we accepted the larger uncertainty.

\begin{figure}[h]
	\includegraphics[width=\columnwidth]{./fe-peak.pdf}
	\caption{Fe-peak (V-Zn) element abundance ratios for standard field stars and Orphan Stream candidates. Color scheme is the same employed for Figure \ref{fig:alpha-fe}.}
	\label{fig:fe-peak}
\end{figure}


\subsubsection{Neutron-capture elements (Sr-Eu)}


\begin{longtable*}{lcrcrrcclcrcrrc}
\caption{\\ Chemical Abundances\label{tab:chemical-abundances}} \tabularnewline
\cline{1-15}
Species & $N$ & $\log\epsilon(X)$ & $\sigma_\epsilon$ & [X/H] & [X/Fe] & $\sigma$ && 
Species & $N$ & $\log\epsilon(X)$ & $\sigma_\epsilon$ & [X/H] & [X/Fe] & $\sigma$ \tabularnewline
\cline{1-15} \tabularnewline
% This works.
\endhead
\hline
\multicolumn{15}{r}{Continued..}
\endfoot
\hline
\endlastfoot

\\
\multicolumn{7}{c}{\textbf{HD 41667}} & \colhead{} & \multicolumn{7}{c}{\textbf{HD 44007}} \\
\cline{1-7} \cline{9-15}

   C (CH)       &   2 &    6.95 &    0.20 & $-$1.48 & $-$0.24 &    0.20 &&
   C (CH)       &   2 &    6.73 &    0.20 & $-$1.70 & $-$0.01 &    0.20 \\
   O \textsc{I} &   1 &    7.82 &    0.00 & $-$0.87 &    0.37 &    0.00 &&
   O \textsc{I} &   1 &    7.48 &    0.00 & $-$1.21 &    0.48 &    0.00 \\
  Na \textsc{I} &   3 &    4.93 &    0.10 & $-$1.31 & $-$0.07 &    0.06 &&
  Na \textsc{I} &   1 &    4.58 &    0.00 & $-$1.66 &    0.03 &    0.00 \\
  Mg \textsc{I} &   8 &    6.73 &    0.23 & $-$0.87 &    0.37 &    0.08 &&
  Mg \textsc{I} &   6 &    6.35 &    0.13 & $-$1.25 &    0.44 &    0.05 \\
  Al \textsc{I} &   4 &    5.25 &    0.10 & $-$1.20 &    0.04 &    0.05 &&
  Al \textsc{I} &   0 & \nodata & \nodata & \nodata & \nodata & \nodata \\
  Si \textsc{I} &   8 &    6.56 &    0.14 & $-$0.95 &    0.29 &    0.05 &&
  Si \textsc{I} &   9 &    6.17 &    0.16 & $-$1.34 &    0.35 &    0.05 \\
   K \textsc{I} &   2 &    4.58 &    0.00 & $-$0.46 &    0.78 &    0.00 &&
   K \textsc{I} &   1 &    4.39 &    0.00 & $-$0.64 &    1.05 &    0.00 \\
  Ca \textsc{I} &  19 &    5.46 &    0.07 & $-$0.88 &    0.36 &    0.01 &&
  Ca \textsc{I} &  22 &    5.11 &    0.10 & $-$1.23 &    0.46 &    0.02 \\
 Sc \textsc{II} &  15 &    2.06 &    0.12 & $-$1.09 &    0.14 &    0.03 &&
 Sc \textsc{II} &  14 &    1.50 &    0.09 & $-$1.65 &    0.04 &    0.02 \\
  Ti \textsc{I} &  26 &    3.82 &    0.18 & $-$1.13 &    0.11 &    0.04 &&
  Ti \textsc{I} &  23 &    3.42 &    0.13 & $-$1.53 &    0.16 &    0.03 \\
 Ti \textsc{II} &  36 &    4.18 &    0.31 & $-$0.77 &    0.47 &    0.05 &&
 Ti \textsc{II} &  40 &    3.65 &    0.19 & $-$1.30 &    0.39 &    0.03 \\
   V \textsc{I} &   5 &    2.83 &    0.14 & $-$1.10 &    0.14 &    0.06 &&
   V \textsc{I} &   4 &    2.27 &    0.04 & $-$1.66 &    0.04 &    0.02 \\
  Cr \textsc{I} &  14 &    4.22 &    0.11 & $-$1.42 & $-$0.18 &    0.03 &&
  Cr \textsc{I} &  16 &    3.78 &    0.07 & $-$1.86 & $-$0.17 &    0.02 \\
 Cr \textsc{II} &   3 &    4.62 &    0.12 & $-$1.02 &    0.22 &    0.07 &&
 Cr \textsc{II} &   3 &    4.12 &    0.04 & $-$1.52 &    0.17 &    0.02 \\
  Mn \textsc{I} &   8 &    3.98 &    0.19 & $-$1.45 & $-$0.21 &    0.07 &&
  Mn \textsc{I} &   9 &    3.38 &    0.15 & $-$2.05 & $-$0.36 &    0.05 \\
  Fe \textsc{I} &  75 &    6.26 &    0.13 & $-$1.24 &    0.00 &    0.01 &&
  Fe \textsc{I} &  76 &    5.81 &    0.12 & $-$1.69 &    0.00 &    0.01 \\
 Fe \textsc{II} &  18 &    6.26 &    0.12 & $-$1.24 &    0.00 &    0.03 &&
 Fe \textsc{II} &  18 &    5.82 &    0.14 & $-$1.68 &    0.01 &    0.03 \\
  Co \textsc{I} &   4 &    3.74 &    0.13 & $-$1.25 & $-$0.01 &    0.06 &&
  Co \textsc{I} &   4 &    3.39 &    0.17 & $-$1.60 &    0.09 &    0.08 \\
  Ni \textsc{I} &  27 &    4.97 &    0.20 & $-$1.25 & $-$0.01 &    0.04 &&
  Ni \textsc{I} &  27 &    4.48 &    0.16 & $-$1.74 & $-$0.05 &    0.03 \\
  Cu \textsc{I} &   1 &    2.76 &    0.00 & $-$1.43 & $-$0.19 &    0.00 &&
  Cu \textsc{I} &   1 &    1.92 &    0.00 & $-$2.27 & $-$0.58 &    0.00 \\
  Zn \textsc{I} &   2 &    3.35 &    0.04 & $-$1.21 &    0.03 &    0.03 &&
  Zn \textsc{I} &   2 &    2.90 &    0.08 & $-$1.66 &    0.03 &    0.06 \\
 Sr \textsc{II} &   1 &    1.70 &    0.00 & $-$1.17 &    0.07 &    0.00 &&
 Sr \textsc{II} &   1 &    1.21 &    0.00 & $-$1.66 &    0.03 &    0.00 \\
  Y \textsc{II} &   3 &    1.03 &    0.12 & $-$1.18 &    0.06 &    0.07 &&
  Y \textsc{II} &   3 &    0.40 &    0.06 & $-$1.81 & $-$0.12 &    0.04 \\
 Ba \textsc{II} &   2 &    0.89 &    0.05 & $-$1.30 & $-$0.06 &    0.03 &&
 Ba \textsc{II} &   2 &    0.47 &    0.06 & $-$1.71 & $-$0.02 &    0.04 \\
 La \textsc{II} &   2 &    0.10 &    0.03 & $-$1.00 &    0.24 &    0.02 &&
 La \textsc{II} &   2 & $-$0.37 &    0.00 & $-$1.47 &    0.22 &    0.00 \\
 Nd \textsc{II} &   7 &    0.48 &    0.08 & $-$0.94 &    0.30 &    0.03 &&
 Nd \textsc{II} &   8 & $-$0.22 &    0.09 & $-$1.64 &    0.05 &    0.03 \\
 Eu \textsc{II} &   1 & $-$0.19 &    0.00 & $-$0.71 &    0.53 &    0.00 &&
 Eu \textsc{II} &   1 & $-$1.12 &    0.00 & $-$1.64 &    0.05 &    0.00 \\


\cline{1-7} \cline{9-15} \\ \\
\multicolumn{7}{c}{\textbf{HD 76932}} && \multicolumn{7}{c}{\textbf{HD 122563}} \\
\cline{1-7} \cline{9-15}

   C (CH)       &   2 &    7.63 &    0.20 & $-$0.79 &    0.15 &    0.20 &&
   C (CH)       &   2 &    5.26 &    0.20 & $-$3.17 & $-$0.27 &    0.20 \\
   O \textsc{I} &   1 &    8.33 &         &$<-$0.36 & $<$0.58 &         &&
   O \textsc{I} &   1 &    6.15 &    0.00 & $-$2.54 &    0.36 &    0.00 \\
  Na \textsc{I} &   3 &    5.44 &    0.04 & $-$0.80 &    0.14 &    0.02 &&
  Na \textsc{I} &   1 & $<$3.48 &         &$<-$2.76 & $<$0.14 &         \\
  Mg \textsc{I} &   8 &    6.95 &    0.18 & $-$0.65 &    0.30 &    0.07 &&
  Mg \textsc{I} &   8 &    5.27 &    0.08 & $-$2.33 &    0.56 &    0.03 \\
  Al \textsc{I} &   4 &    5.57 &    0.08 & $-$0.88 &    0.06 &    0.04 &&
  Al \textsc{I} &   1 & $<$4.83 &         &$<-$1.62 & $<$1.28 &         \\
  Si \textsc{I} &   9 &    6.80 &    0.18 & $-$0.71 &    0.24 &    0.06 &&
  Si \textsc{I} &   1 &    5.23 &    0.00 & $-$2.28 &    0.62 &    0.00 \\
   K \textsc{I} &   2 &    4.97 &    0.06 & $-$0.06 &    0.89 &    0.05 &&
   K \textsc{I} &   1 &    2.76 &    0.00 & $-$2.27 &    0.63 &    0.00 \\
  Ca \textsc{I} &  24 &    5.70 &    0.10 & $-$0.64 &    0.30 &    0.02 &&
  Ca \textsc{I} &  21 &    3.85 &    0.09 & $-$2.49 &    0.40 &    0.02 \\
 Sc \textsc{II} &  16 &    2.39 &    0.10 & $-$0.76 &    0.19 &    0.03 &&
 Sc \textsc{II} &  15 &    0.18 &    0.07 & $-$2.97 & $-$0.07 &    0.02 \\
  Ti \textsc{I} &  23 &    4.28 &    0.19 & $-$0.67 &    0.27 &    0.04 &&
  Ti \textsc{I} &  22 &    2.12 &    0.10 & $-$2.83 &    0.06 &    0.02 \\
 Ti \textsc{II} &  39 &    4.38 &    0.13 & $-$0.57 &    0.37 &    0.02 &&
 Ti \textsc{II} &  42 &    2.24 &    0.10 & $-$2.71 &    0.18 &    0.02 \\
   V \textsc{I} &   4 &    3.25 &    0.16 & $-$0.68 &    0.26 &    0.08 &&
   V \textsc{I} &   1 &    0.81 &    0.00 & $-$3.12 & $-$0.22 &    0.00 \\
  Cr \textsc{I} &  20 &    4.57 &    0.10 & $-$1.07 & $-$0.12 &    0.02 &&
  Cr \textsc{I} &  16 &    2.42 &    0.11 & $-$3.22 & $-$0.32 &    0.03 \\
 Cr \textsc{II} &   4 &    4.91 &    0.08 & $-$0.73 &    0.21 &    0.04 &&
 Cr \textsc{II} &   3 &    2.83 &    0.03 & $-$2.81 &    0.09 &    0.02 \\
  Mn \textsc{I} &   9 &    4.32 &    0.10 & $-$1.11 & $-$0.17 &    0.03 &&
  Mn \textsc{I} &   7 &    2.15 &    0.05 & $-$3.28 & $-$0.39 &    0.02 \\
  Fe \textsc{I} &  97 &    6.56 &    0.11 & $-$0.94 &    0.00 &    0.01 &&
  Fe \textsc{I} & 166 &    4.60 &    0.11 & $-$2.90 &    0.00 &    0.01 \\
 Fe \textsc{II} &  20 &    6.56 &    0.14 & $-$0.94 &    0.00 &    0.03 &&
 Fe \textsc{II} &  24 &    4.61 &    0.16 & $-$2.89 &    0.01 &    0.03 \\
  Co \textsc{I} &   4 &    4.11 &    0.12 & $-$0.88 &    0.06 &    0.06 &&
  Co \textsc{I} &   5 &    2.25 &    0.14 & $-$2.74 &    0.16 &    0.06 \\
  Ni \textsc{I} &  27 &    5.31 &    0.13 & $-$0.91 &    0.03 &    0.03 &&
  Ni \textsc{I} &  19 &    3.49 &    0.10 & $-$2.73 &    0.16 &    0.02 \\
  Cu \textsc{I} &   1 &    2.96 &    0.00 & $-$1.23 & $-$0.29 &    0.00 &&
  Cu \textsc{I} &   1 & $<$0.10 &         &$<-$4.19 &$<-$1.19 &         \\
  Zn \textsc{I} &   2 &    3.71 &    0.01 & $-$0.84 &    0.10 &    0.01 &&
  Zn \textsc{I} &   2 &    1.84 &    0.06 & $-$2.72 &    0.18 &    0.04 \\
 Sr \textsc{II} &   1 &    2.01 &    0.00 & $-$0.86 &    0.08 &    0.00 &&
 Sr \textsc{II} &   1 & $-$0.59 &    0.00 & $-$3.46 & $-$0.56 &    0.00 \\
  Y \textsc{II} &   3 &    1.41 &    0.17 & $-$0.80 &    0.14 &    0.10 &&
  Y \textsc{II} &   1 & $-$0.81 &    0.00 & $-$3.02 & $-$0.12 &    0.00 \\
 Ba \textsc{II} &   2 &    1.36 &    0.05 & $-$0.82 &    0.12 &    0.04 &&
 Ba \textsc{II} &   2 & $-$1.92 &    0.06 & $-$4.10 & $-$1.20 &    0.05 \\
 La \textsc{II} &   1 &    0.69 &    0.00 & $-$0.41 &    0.53 &    0.00 &&
 La \textsc{II} &   1 &$<-$1.51 &         &$<-$2.61 & $<$0.29 &         \\
 Ce \textsc{II} &   1 & $<$0.70 &         &$<-$0.88 & $<$0.06 &         &&
 Ce \textsc{II} &   1 &$<-$1.53 &         &$<-$3.11 &$<-$0.21 &         \\
 Nd \textsc{II} &   3 &    0.82 &    0.11 & $-$0.60 &    0.34 &    0.07 &&
 Nd \textsc{II} &   1 &$<-$1.83 &         &$<-$3.25 &$<-$0.35 &         \\
 Eu \textsc{II} &   1 &$<-$0.27 &         &$<-$0.79 & $<$0.15 &         &&
 Eu \textsc{II} &   1 &$<-$1.72 &         &$<-$2.24 & $<$0.66 &         \\


\cline{1-7} \cline{9-15} \\ \\
\multicolumn{7}{c}{\textbf{HD 136316}} && \multicolumn{7}{c}{\textbf{HD 141531}} \\
\cline{1-7} \cline{9-15}

   C (CH)       &   2 &    6.05 &    0.20 & $-$2.38 & $-$0.47 &    0.20 &&
   C (CH)       &   2 &    6.31 &    0.20 & $-$2.12 & $-$0.42 &    0.20 \\
   O \textsc{I} &   1 &    7.16 &    0.00 & $-$1.53 &    0.38 &    0.00 &&
   O \textsc{I} &   2 &    7.30 &    0.00 & $-$1.38 &    0.32 &    0.00 \\
  Na \textsc{I} &   2 &    4.21 &    0.08 & $-$2.04 & $-$0.12 &    0.05 &&
  Na \textsc{I} &   3 &    4.38 &    0.07 & $-$1.86 & $-$0.15 &    0.04 \\
  Mg \textsc{I} &  10 &    6.21 &    0.33 & $-$1.39 &    0.52 &    0.10 &&
  Mg \textsc{I} &  10 &    6.36 &    0.33 & $-$1.24 &    0.46 &    0.10 \\
  Al \textsc{I} &   1 & $<$4.58 &         &$<-$1.87 & $<$0.04 &         &&
  Al \textsc{I} &   1 &    4.80 &    0.00 & $-$1.65 &    0.05 &    0.00 \\
  Si \textsc{I} &   9 &    5.97 &    0.18 & $-$1.54 &    0.37 &    0.06 &&
  Si \textsc{I} &   8 &    6.09 &    0.16 & $-$1.42 &    0.28 &    0.06 \\
   K \textsc{I} &   2 &    3.88 &    0.03 & $-$1.15 &    0.76 &    0.02 &&
   K \textsc{I} &   2 &    4.00 &    0.04 & $-$1.04 &    0.67 &    0.03 \\
  Ca \textsc{I} &  21 &    4.79 &    0.09 & $-$1.55 &    0.36 &    0.02 &&
  Ca \textsc{I} &  23 &    5.02 &    0.16 & $-$1.32 &    0.38 &    0.03 \\
 Sc \textsc{II} &  16 &    1.29 &    0.08 & $-$1.86 &    0.06 &    0.02 &&
 Sc \textsc{II} &  16 &    1.46 &    0.09 & $-$1.69 &    0.01 &    0.02 \\
  Ti \textsc{I} &  29 &    3.12 &    0.22 & $-$1.83 &    0.08 &    0.04 &&
  Ti \textsc{I} &  30 &    3.26 &    0.24 & $-$1.69 &    0.01 &    0.04 \\
 Ti \textsc{II} &  42 &    3.52 &    0.23 & $-$1.43 &    0.49 &    0.04 &&
 Ti \textsc{II} &  40 &    3.67 &    0.26 & $-$1.28 &    0.42 &    0.04 \\
   V \textsc{I} &   8 &    2.47 &    0.97 & $-$1.46 &    0.45 &    0.34 &&
   V \textsc{I} &   7 &    2.19 &    0.14 & $-$1.74 & $-$0.04 &    0.05 \\
  Cr \textsc{I} &  21 &    3.54 &    0.25 & $-$2.10 & $-$0.18 &    0.06 &&
  Cr \textsc{I} &  20 &    3.70 &    0.26 & $-$1.94 & $-$0.24 &    0.06 \\
 Cr \textsc{II} &   2 &    3.85 &    0.02 & $-$1.79 &    0.12 &    0.01 &&
 Cr \textsc{II} &   2 &    4.20 &    0.10 & $-$1.44 &    0.26 &    0.07 \\
  Mn \textsc{I} &  10 &    3.24 &    0.26 & $-$2.19 & $-$0.28 &    0.08 &&
  Mn \textsc{I} &   9 &    3.48 &    0.29 & $-$1.95 & $-$0.25 &    0.10 \\
  Fe \textsc{I} & 101 &    5.59 &    0.13 & $-$1.91 &    0.00 &    0.01 &&
  Fe \textsc{I} &  85 &    5.80 &    0.13 & $-$1.70 &    0.00 &    0.01 \\
 Fe \textsc{II} &  18 &    5.61 &    0.15 & $-$1.89 &    0.03 &    0.03 &&
 Fe \textsc{II} &  18 &    5.78 &    0.11 & $-$1.72 & $-$0.01 &    0.03 \\
  Co \textsc{I} &   4 &    3.12 &    0.12 & $-$1.87 &    0.04 &    0.06 &&
  Co \textsc{I} &   6 &    3.23 &    0.15 & $-$1.76 & $-$0.06 &    0.06 \\
  Ni \textsc{I} &  26 &    4.24 &    0.13 & $-$1.98 & $-$0.07 &    0.03 &&
  Ni \textsc{I} &  27 &    4.44 &    0.16 & $-$1.78 & $-$0.07 &    0.03 \\
  Cu \textsc{I} &   1 &    1.74 &    0.00 & $-$2.45 & $-$0.54 &    0.00 &&
  Cu \textsc{I} &   1 &    2.06 &    0.00 & $-$2.13 & $-$0.43 &    0.00 \\
  Zn \textsc{I} &   2 &    2.67 &    0.10 & $-$1.88 &    0.03 &    0.07 &&
  Zn \textsc{I} &   2 &    2.75 &    0.04 & $-$1.81 & $-$0.11 &    0.03 \\
 Sr \textsc{II} &   1 &    0.94 &    0.00 & $-$1.93 & $-$0.02 &    0.00 &&
 Sr \textsc{II} &   1 &    1.13 &    0.00 & $-$1.74 & $-$0.04 &    0.00 \\
  Y \textsc{II} &   3 &    0.17 &    0.09 & $-$2.04 & $-$0.13 &    0.05 &&
  Y \textsc{II} &   3 &    0.30 &    0.11 & $-$1.91 & $-$0.21 &    0.06 \\
 Ba \textsc{II} &   2 &    0.26 &    0.08 & $-$1.93 & $-$0.01 &    0.05 &&
 Ba \textsc{II} &   2 &    0.35 &    0.05 & $-$1.83 & $-$0.12 &    0.04 \\
 La \textsc{II} &   2 & $-$0.69 &    0.02 & $-$1.79 &    0.13 &    0.02 &&
 La \textsc{II} &   2 & $-$0.59 &    0.07 & $-$1.70 &    0.01 &    0.05 \\
 Ce \textsc{II} &   2 & $-$0.29 &    0.22 & $-$1.87 &    0.04 &    0.16 &&
 Ce \textsc{II} &   1 &    0.15 &    0.00 & $-$1.43 &    0.27 &    0.00 \\
 Nd \textsc{II} &   8 & $-$0.36 &    0.05 & $-$1.78 &    0.14 &    0.02 &&
 Nd \textsc{II} &   7 & $-$0.24 &    0.07 & $-$1.66 &    0.04 &    0.03 \\
 Eu \textsc{II} &   1 & $-$1.08 &    0.00 & $-$1.60 &    0.31 &    0.00 &&
 Eu \textsc{II} &   1 & $-$0.97 &    0.00 & $-$1.49 &    0.21 &    0.00 \\


\cline{1-7} \cline{9-15} \\ \\
\multicolumn{7}{c}{\textbf{HD 142948}} && \multicolumn{7}{c}{\textbf{OSS 3}} \\
\cline{1-7} \cline{9-15}


   C (CH)       &   2 &    7.75 &    0.20 & $-$0.68 &    0.03 &    0.20 &&
   C (CH)       &   2 &    7.54 &    0.20 & $-$0.89 &    0.00 &    0.20 \\
   O \textsc{I} &   2 &    8.47 &    0.03 & $-$0.21 &    0.50 &    0.02 &&
   O \textsc{I} &   1 & $<$8.43 &         &$<-$0.25 & $<$0.54 &         \\
  Na \textsc{I} &   3 &    5.76 &    0.13 & $-$0.48 &    0.23 &    0.08 &&
  Na \textsc{I} &   2 &    5.22 &    0.01 & $-$1.02 & $-$0.16 &    0.01 \\
  Mg \textsc{I} &   7 &    7.24 &    0.18 & $-$0.36 &    0.35 &    0.07 &&
  Mg \textsc{I} &   7 &    6.75 &    0.17 & $-$0.85 &    0.01 &    0.07 \\
  Al \textsc{I} &   4 &    6.00 &    0.08 & $-$0.45 &    0.27 &    0.04 &&
  Al \textsc{I} &   2 &    5.27 &    0.11 & $-$1.18 & $-$0.32 &    0.08 \\
  Si \textsc{I} &  10 &    7.11 &    0.18 & $-$0.40 &    0.31 &    0.06 &&
  Si \textsc{I} &   8 &    6.72 &    0.17 & $-$0.79 &    0.07 &    0.06 \\
   K \textsc{I} &   1 &    5.08 &    0.00 &    0.05 &    0.76 &    0.00 &&
   K \textsc{I} &   1 &    4.84 &    0.00 & $-$0.19 &    0.67 &    0.00 \\
  Ca \textsc{I} &  19 &    5.87 &    0.14 & $-$0.47 &    0.24 &    0.03 &&
  Ca \textsc{I} &  23 &    5.73 &    0.16 & $-$0.61 &    0.26 &    0.03 \\
 Sc \textsc{II} &  16 &    2.73 &    0.26 & $-$0.42 &    0.30 &    0.06 &&
 Sc \textsc{II} &  13 &    2.28 &    0.13 & $-$0.87 &    0.00 &    0.04 \\
  Ti \textsc{I} &  26 &    4.39 &    0.22 & $-$0.56 &    0.16 &    0.04 &&
  Ti \textsc{I} &  25 &    4.33 &    0.33 & $-$0.62 &    0.24 &    0.07 \\
 Ti \textsc{II} &  32 &    4.60 &    0.21 & $-$0.35 &    0.36 &    0.04 &&
 Ti \textsc{II} &  38 &    4.39 &    0.25 & $-$0.56 &    0.31 &    0.04 \\
   V \textsc{I} &   5 &    3.47 &    0.20 & $-$0.46 &    0.26 &    0.09 &&
   V \textsc{I} &   3 &    3.01 &    0.11 & $-$0.92 & $-$0.06 &    0.06 \\
  Cr \textsc{I} &  18 &    4.85 &    0.30 & $-$0.79 & $-$0.08 &    0.07 &&
  Cr \textsc{I} &  22 &    4.66 &    0.27 & $-$0.98 & $-$0.11 &    0.06 \\
 Cr \textsc{II} &   3 &    5.09 &    0.13 & $-$0.55 &    0.16 &    0.08 &&
 Cr \textsc{II} &   3 &    4.89 &    0.26 & $-$0.75 &    0.11 &    0.15 \\
  Mn \textsc{I} &   9 &    4.70 &    0.21 & $-$0.73 & $-$0.02 &    0.07 &&
  Mn \textsc{I} &  10 &    4.45 &    0.36 & $-$0.98 & $-$0.12 &    0.11 \\
  Fe \textsc{I} &  69 &    6.79 &    0.14 & $-$0.71 &    0.00 &    0.02 &&
  Fe \textsc{I} &  97 &    6.64 &    0.20 & $-$0.86 &    0.00 &    0.02 \\
 Fe \textsc{II} &  17 &    6.79 &    0.19 & $-$0.71 &    0.00 &    0.05 &&
 Fe \textsc{II} &  27 &    6.64 &    0.24 & $-$0.86 &    0.00 &    0.05 \\
  Co \textsc{I} &   5 &    4.41 &    0.11 & $-$0.58 &    0.13 &    0.05 &&
  Co \textsc{I} &   4 &    4.26 &    0.26 & $-$0.73 &    0.14 &    0.13 \\
  Ni \textsc{I} &  26 &    5.61 &    0.22 & $-$0.61 &    0.10 &    0.04 &&
  Ni \textsc{I} &  25 &    5.33 &    0.24 & $-$0.89 & $-$0.03 &    0.05 \\
  Cu \textsc{I} &   1 &    3.74 &    0.00 & $-$0.45 &    0.26 &    0.00 &&
  Cu \textsc{I} &   1 &    3.05 &    0.00 & $-$1.14 & $-$0.28 &    0.00 \\
  Zn \textsc{I} &   2 &    3.87 &    0.14 & $-$0.69 &    0.02 &    0.10 &&
  Zn \textsc{I} &   2 &    3.67 &    0.15 & $-$0.89 & $-$0.03 &    0.10 \\
 Sr \textsc{II} &   1 &    2.32 &    0.00 & $-$0.55 &    0.16 &    0.00 &&
 Sr \textsc{II} &   1 &    1.95 &    0.00 & $-$0.92 & $-$0.06 &    0.00 \\
  Y \textsc{II} &   3 &    1.62 &    0.24 & $-$0.59 &    0.12 &    0.14 &&
  Y \textsc{II} &   2 &    1.27 &    0.35 & $-$0.94 & $-$0.07 &    0.25 \\
 Ba \textsc{II} &   2 &    1.29 &    0.01 & $-$0.90 & $-$0.18 &    0.00 &&
 Ba \textsc{II} &   2 &    1.51 &    0.13 & $-$0.67 &    0.19 &    0.09 \\
 La \textsc{II} &   2 &    0.51 &    0.10 & $-$0.59 &    0.12 &    0.07 &&
 La \textsc{II} &   1 & $<$0.59 &         &$<-$1.51 & $<$0.38 &         \\
 Nd \textsc{II} &   4 &    0.91 &    0.07 & $-$0.51 &    0.20 &    0.03 &&
 Nd \textsc{II} &   1 &    1.46 &    0.00 &    0.04 &    0.90 &    0.00 \\
 Eu \textsc{II} &   1 &    0.15 &    0.00 & $-$0.37 &    0.34 &    0.00 &&
 Eu \textsc{II} &   1 &$<-$0.09 &         &$<-$0.61 & $<$0.28 &         \\

\cline{1-7} \cline{9-15} \\ \\
\multicolumn{7}{c}{\textbf{OSS 6}} && \multicolumn{7}{c}{\textbf{OSS 8}} \\
\cline{1-7} \cline{9-15}

%   O \textsc{I} &   1 &    6.80 &    0.00 & $-$1.89 & $-$0.19 &    0.00 &&
%  Al \textsc{I} &   1 &    4.84 &    0.00 & $-$1.61 &    0.09 &    0.00 &&

   C (CH)       &   2 &    6.39 &    0.20 & $-$2.04 & $-$0.37 &    0.20 &&
   C (CH)       &   2 &    6.85 &    0.20 & $-$1.58 &    0.00 &    0.20 \\
   O \textsc{I} &   1 & $<$7.08 &         &$<-$1.61 & $<$0.06 &         &&
   O \textsc{I} &   1 & $<$7.46 &         &$<-$1.23 & $<$0.35 &         \\
  Na \textsc{I} &   1 &    4.27 &    0.00 & $-$1.97 & $-$0.27 &    0.00 &&
  Na \textsc{I} &   1 &    4.36 &    0.00 & $-$1.88 & $-$0.26 &    0.00 \\
  Mg \textsc{I} &   6 &    5.99 &    0.20 & $-$1.61 &    0.09 &    0.08 &&
  Mg \textsc{I} &  11 &    6.07 &    0.41 & $-$1.53 &    0.09 &    0.12 \\
  Al \textsc{I} &   1 & $<$4.85 &         &$<-$1.60 & $<$0.07 &         &&
  Al \textsc{I} &   1 & $<$5.18 &         &$<-$1.27 & $<$0.31 &         \\
  Si \textsc{I} &   3 &    5.89 &    0.10 & $-$1.62 &    0.07 &    0.06 &&
  Si \textsc{I} &   3 &    6.14 &    0.21 & $-$1.37 &    0.25 &    0.12 \\
   K \textsc{I} &   1 &    3.96 &    0.00 & $-$1.07 &    0.63 &    0.00 &&
   K \textsc{I} &   1 &    4.24 &    0.00 & $-$0.79 &    0.83 &    0.00 \\
  Ca \textsc{I} &  24 &    4.88 &    0.24 & $-$1.46 &    0.24 &    0.05 &&
  Ca \textsc{I} &  23 &    5.04 &    0.15 & $-$1.30 &    0.32 &    0.03 \\
 Sc \textsc{II} &  16 &    1.25 &    0.17 & $-$1.90 & $-$0.20 &    0.04 &&
 Sc \textsc{II} &  16 &    1.47 &    0.15 & $-$1.68 & $-$0.06 &    0.04 \\
  Ti \textsc{I} &  23 &    3.17 &    0.19 & $-$1.78 & $-$0.09 &    0.04 &&
  Ti \textsc{I} &  20 &    3.31 &    0.11 & $-$1.64 & $-$0.02 &    0.02 \\
 Ti \textsc{II} &  39 &    3.51 &    0.21 & $-$1.44 &    0.25 &    0.03 &&
 Ti \textsc{II} &  39 &    3.59 &    0.24 & $-$1.36 &    0.26 &    0.04 \\
   V \textsc{I} &   2 &    1.78 &    0.09 & $-$2.15 & $-$0.45 &    0.07 &&
   V \textsc{I} &   2 &    2.29 &    0.18 & $-$1.64 & $-$0.02 &    0.13 \\
  Cr \textsc{I} &  20 &    3.81 &    0.14 & $-$1.83 & $-$0.14 &    0.03 &&
  Cr \textsc{I} &  17 &    3.82 &    0.12 & $-$1.82 & $-$0.19 &    0.03 \\
 Cr \textsc{II} &   2 &    4.19 &    0.07 & $-$1.45 &    0.24 &    0.05 &&
 Cr \textsc{II} &   3 &    4.23 &    0.04 & $-$1.41 &    0.21 &    0.02 \\
  Mn \textsc{I} &  10 &    3.64 &    0.38 & $-$1.79 & $-$0.09 &    0.12 &&
  Mn \textsc{I} &   5 &    3.53 &    0.18 & $-$1.90 & $-$0.28 &    0.08 \\
  Fe \textsc{I} &  69 &    5.80 &    0.15 & $-$1.70 &    0.00 &    0.02 &&
  Fe \textsc{I} &  60 &    5.88 &    0.15 & $-$1.62 &    0.00 &    0.02 \\
 Fe \textsc{II} &  15 &    5.81 &    0.14 & $-$1.69 &    0.01 &    0.04 &&
 Fe \textsc{II} &  12 &    5.88 &    0.16 & $-$1.62 &    0.00 &    0.05 \\
  Co \textsc{I} &   3 &    3.17 &    0.35 & $-$1.82 & $-$0.12 &    0.20 &&
  Co \textsc{I} &   4 &    3.35 &    0.14 & $-$1.64 & $-$0.02 &    0.07 \\
  Ni \textsc{I} &  23 &    4.42 &    0.19 & $-$1.80 & $-$0.10 &    0.04 &&
  Ni \textsc{I} &  21 &    4.54 &    0.16 & $-$1.68 & $-$0.06 &    0.04 \\
  Cu \textsc{I} &   1 &    1.49 &    0.00 & $-$2.70 & $-$1.00 &    0.00 &&
  Cu \textsc{I} &   1 &    1.68 &    0.00 & $-$2.51 & $-$0.89 &    0.00 \\
  Zn \textsc{I} &   2 &    2.51 &    0.14 & $-$2.04 & $-$0.35 &    0.10 &&
  Zn \textsc{I} &   1 &    2.89 &    0.00 & $-$1.67 & $-$0.05 &    0.00 \\
 Sr \textsc{II} &   1 &    0.83 &    0.00 & $-$2.04 & $-$0.34 &    0.00 &&
 Sr \textsc{II} &   1 &    0.79 &    0.00 & $-$2.08 & $-$0.46 &    0.00 \\
  Y \textsc{II} &   1 &$<-$0.59 &         &$<-$2.77 &$<-$1.10 &         &&
  Y \textsc{II} &   1 &$<-$0.41 &         &$<-$2.60 &$<-$1.00 &         \\
 Ba \textsc{II} &   1 &    0.20 &         & $-$1.98 & $-$0.28 &         &&
 Ba \textsc{II} &   1 & $-$0.24 &         & $-$2.42 & $-$0.80 &         \\
 La \textsc{II} &   1 &$<-$0.47 &         &$<-$1.57 & $<$0.10 &         &&
 La \textsc{II} &   1 &   -0.12 &    0.00 & $-$1.22 &    0.40 &    0.00 \\
 Nd \textsc{II} &   2 &   -0.64 &    0.08 & $-$2.06 & $-$0.36 &    0.06 &&
 Nd \textsc{II} &   1 &    0.08 &    0.00 & $-$1.34 &    0.28 &    0.00 \\
 Eu \textsc{II} &   1 &$<-$1.25 &         &$<-$1.77 &$<-$0.10 &         &&
 Eu \textsc{II} &   1 &$<-$0.62 &         &$<-$1.14 & $<$0.44 &         \\

\cline{1-7} \cline{9-15} \\ \\
\multicolumn{7}{c}{\textbf{OSS 14}} && \multicolumn{7}{c}{\textbf{OSS 18}} \\
\cline{1-7} \cline{9-15}



   C (CH)       &   2 &    5.85 &    0.20 & $-$2.58 &    0.08 &    0.20 &&
   C (CH)       &   1 &    7.57 &    0.20 & $-$0.86 & $-$0.10 &    0.20 \\
   O \textsc{I} &   1 & $<$6.83 &         &$<-$1.86 & $<$0.80 &         &&
   O \textsc{I} &   0 & \nodata & \nodata & \nodata & \nodata & \nodata \\
  Na \textsc{I} &   1 & $<$4.77 &         &$<-$1.47 & $<$1.19 &         &&
  Na \textsc{I} &   1 &    5.37 &    0.00 & $-$0.87 & $-$0.25 &    0.00 \\
  Mg \textsc{I} &   8 &    5.19 &    0.21 & $-$2.41 &    0.25 &    0.07 &&
  Mg \textsc{I} &   5 &    7.01 &    0.25 & $-$0.59 &    0.03 &    0.11 \\
  Al \textsc{I} &   1 &    5.29 &    0.00 & $-$1.16 &    1.51 &    0.00 &&
  Al \textsc{I} &   1 & $<$5.66 &         &$<-$0.79 &$<-$0.03 &         \\
  Si \textsc{I} &   2 &    5.50 &    0.03 & $-$2.01 &    0.66 &    0.02 &&
  Si \textsc{I} &   5 &    6.98 &    0.28 & $-$0.53 &    0.09 &    0.13 \\
   K \textsc{I} &   1 &    2.66 &    0.00 & $-$2.37 &    0.30 &    0.00 &&
   K \textsc{I} &   1 &    4.91 &    0.00 & $-$0.12 &    0.50 &    0.00 \\
  Ca \textsc{I} &  16 &    3.89 &    0.15 & $-$2.45 &    0.21 &    0.04 &&
  Ca \textsc{I} &  22 &    5.89 &    0.24 & $-$0.45 &    0.17 &    0.05 \\
 Sc \textsc{II} &   9 &    0.13 &    0.13 & $-$3.02 & $-$0.36 &    0.04 &&
 Sc \textsc{II} &  11 &    2.64 &    0.18 & $-$0.51 &    0.11 &    0.05 \\
  Ti \textsc{I} &  11 &    2.24 &    0.12 & $-$2.71 & $-$0.05 &    0.04 &&
  Ti \textsc{I} &  20 &    4.37 &    0.33 & $-$0.58 &    0.04 &    0.07 \\
 Ti \textsc{II} &  34 &    2.36 &    0.19 & $-$2.59 &    0.08 &    0.03 &&
 Ti \textsc{II} &  28 &    4.54 &    0.43 & $-$0.41 &    0.21 &    0.08 \\
   V \textsc{I} &   1 & $<$1.61 &         &$<-$2.32 & $<$0.34 &         &&
   V \textsc{I} &   4 &    3.26 &    0.01 & $-$0.67 & $-$0.05 &    0.00 \\
  Cr \textsc{I} &  11 &    2.59 &    0.26 & $-$3.05 & $-$0.38 &    0.08 &&
  Cr \textsc{I} &  18 &    4.83 &    0.28 & $-$0.81 & $-$0.19 &    0.06 \\
 Cr \textsc{II} &   1 &    3.06 &    0.00 & $-$2.58 &    0.09 &    0.00 &&
 Cr \textsc{II} &   1 &    5.29 &    0.00 & $-$0.35 &    0.27 &    0.00 \\
  Mn \textsc{I} &   4 &    2.20 &    0.20 & $-$3.23 & $-$0.56 &    0.10 &&
  Mn \textsc{I} &   4 &    4.47 &    0.31 & $-$0.96 & $-$0.33 &    0.16 \\
  Fe \textsc{I} & 133 &    4.83 &    0.20 & $-$2.67 &    0.00 &    0.02 &&
  Fe \textsc{I} &  48 &    6.88 &    0.21 & $-$0.62 &    0.00 &    0.03 \\
 Fe \textsc{II} &  16 &    4.83 &    0.16 & $-$2.67 & $-$0.01 &    0.04 &&
 Fe \textsc{II} &  12 &    6.88 &    0.25 & $-$0.62 &    0.00 &    0.07 \\
  Co \textsc{I} &   3 &    2.32 &    0.48 & $-$2.67 & $-$0.01 &    0.28 &&
  Co \textsc{I} &   3 &    4.65 &    0.41 & $-$0.34 &    0.28 &    0.24 \\
  Ni \textsc{I} &   5 &    3.51 &    0.07 & $-$2.71 & $-$0.04 &    0.03 &&
  Ni \textsc{I} &  18 &    5.46 &    0.26 & $-$0.76 & $-$0.14 &    0.06 \\
  Cu \textsc{I} &   1 & $<$0.77 &         &$<-$3.42 &$<-$0.75 &         &&
  Cu \textsc{I} &   1 &    3.14 &    0.00 & $-$1.05 & $-$0.43 &    0.00 \\
  Zn \textsc{I} &   1 & $<$1.92 &         &$<-$2.64 & $<$0.02 &         &&
  Zn \textsc{I} &   1 &    3.87 &    0.00 & $-$0.69 & $-$0.07 &    0.00 \\
 Sr \textsc{II} &   1 & $-$0.22 &    0.00 & $-$3.89 & $-$0.42 &    0.00 &&
 Sr \textsc{II} &   1 &    2.31 &    0.00 & $-$0.56 &    0.06 &    0.00 \\
  Y \textsc{II} &   1 &$<-$0.86 &         &$<-$3.10 &$<-$0.40 &         &&
  Y \textsc{II} &   1 &    1.92 &         & $-$0.29 &    0.33 &         \\
 Ba \textsc{II} &   2 & $-$1.87 &    0.03 & $-$4.02 & $-$1.35 &    0.02 &&
 Ba \textsc{II} &   2 &    1.76 &    0.05 & $-$0.42 &    0.20 &    0.04 \\
 La \textsc{II} &   1 &$<-$0.91 &         &$<-$2.01 & $<$0.65 &         &&
 La \textsc{II} &   1 &    1.29 &    0.00 &    0.19 &    0.81 &    0.00 \\
 Nd \textsc{II} &   1 &$<-$0.56 &         &$<-$1.98 & $<$0.68 &         &&
 Nd \textsc{II} &   4 &    1.74 &    0.37 &    0.32 &    0.94 &    0.18 \\
 Eu \textsc{II} &   1 &$<-$1.14 &         &$<-$1.66 &  $<$1.0 &         &&
 Eu \textsc{II} &   1 &    0.72 &    0.00 &    0.20 &    0.82 &    0.00 \\
\end{longtable*}

 
 

\begin{deluxetable*}{lcccccc}
\tablecolumns{1}
\tabletypesize{\scriptsize}
\tablecaption{Abundance Uncertainties Due to Stellar Parameters\label{tab:chemical-abundance-uncertainties}}
\tablehead{
 & &  & & & \multicolumn{2}{c}{\textbf{Total Uncertainty}} \\
 \cline{6-7}
	\colhead{Species} &
	\colhead{$T_{\rm eff}+\sigma(T_{\rm eff})\,{\rm K}$} &
	\colhead{$\log{g}+\sigma(\log{g})\,{\rm dex}$} &
	\colhead{${v_t}+\sigma(v_t)$\,km s$^{-1}$} &
	\colhead{$Max(0.10, S.D.)/\sqrt(N)$} & 
	\colhead{[X/H]} &
	\colhead{[X/Fe]} \\
	 & $\Delta$abundance & $\Delta$abundance & $\Delta$abundance & (dex) & (dex) & (dex)
 }
\startdata
\\
\multicolumn{7}{c}{\textbf{OSS 3}} \\
\hline
Na I   & 0.05    & 0.00    & 0.01    & 0.07    & 0.09    & 0.09 \\
Mg I   & 0.05    & 0.02    & 0.03    & 0.06    & 0.09    & 0.18 \\
Al I   & 0.04    & 0.00    & 0.00    & 0.08    & 0.09    & 0.10 \\
Si I   & 0.03    & 0.01    & 0.01    & 0.06    & 0.07    & 0.11 \\
K I    & 0.06    & 0.04    & 0.05    & 0.10    & 0.13    & 0.26 \\
Ca I   & 0.06    & 0.03    & 0.04    & 0.03    & 0.08    & 0.24 \\
Sc II  & 0.02    & 0.05    & 0.01    & 0.04    & 0.06    & 0.30 \\
Ti I   & 0.11    & 0.01    & 0.06    & 0.07    & 0.15    & 0.11 \\
Ti II  & 0.02    & 0.04    & 0.04    & 0.04    & 0.07    & 0.27 \\
V I    & 0.11    & 0.01    & 0.03    & 0.06    & 0.13    & 0.12 \\
Cr I   & 0.11    & 0.02    & 0.06    & 0.06    & 0.14    & 0.19 \\
Cr II  & 0.01    & 0.05    & 0.02    & 0.15    & 0.16    & 0.26 \\
Mn I   & 0.09    & 0.03    & 0.04    & 0.11    & 0.15    & 0.22 \\
Fe I   & 0.08    & 0.00    & 0.04    & 0.02    & 0.09    & \nodata \\
Fe II  & 0.01    & 0.05    & 0.01    & 0.05    & 0.07    & \nodata \\
Co I   & 0.11    & 0.01    & 0.08    & 0.12    & 0.19    & 0.01 \\
Ni I   & 0.07    & 0.01    & 0.04    & 0.05    & 0.09    & 0.13 \\
Cu I   & 0.11    & 0.01    & 0.04    & 0.10    & 0.15    & 0.10 \\
Zn I   & 0.01    & 0.03    & 0.03    & 0.10    & 0.11    & 0.21 \\
Sr II  & 0.03    & 0.00    & 0.01    & 0.10    & 0.10    & 0.11 \\
Y II   & 0.02    & 0.05    & 0.01    & 0.25    & 0.26    & 0.17 \\
Ba II  & 0.02    & 0.02    & 0.05    & 0.08    & 0.10    & 0.17 \\
Nd II  & 0.05    & 0.05    & 0.02    & 0.10    & 0.12    & 0.30
\enddata
\tablenotetext{}{Table \ref{tab:chemical-abundance-uncertainties} is published for all program stars in the electronic edition. A portion is shown here for guidance regarding its form and content.}
\end{deluxetable*}


\section{Discussion}
\label{sec:discussion}

\subsection{Stream Membership}

Given the low surface brightness of the Orphan Stream, separating true members from halo interlopers can be particularly challenging. In this study we have observed high, medium, and low-probability stream members as described in \citet{casey;et-al_2013a}. Before inferring any properties of the undiscovered parent satellite from our sample, we must examine whether our targets are likely to be true disrupted Orphan Stream members.

When compared against the mean of our high-priority targets, the velocities of the low- and medium-probability candidates are +6.2 and $-23.7$\,km s$^{-1}$ different, respectively. Given this region of the stream has a low intrinsic velocity dispersion \citep{newberg;et-al_2010,casey;et-al_2013a}, the significantly lower velocity of the medium-probability target, OSS 18, is intriguing. Perhaps more concerning is that the low- and medium-probability candidates are markedly more metal-rich than the high-priority targets: [Fe/H] = $-0.86$ and $-0.62$\,dex for OSS 3 and 18, a difference of $+0.45$ and $+$0.28\,dex respectively from the low-resolution spectroscopic determinations. Spectroscopic studies suggest the stream is significantly more metal-poor than [Fe/H] $\sim -0.8$\,dex \citep{newberg;et-al_2010,casey;et-al_2013a,sesar;et-al_2013}. The metallicities found from low-resolution spectroscopy by \citet{casey;et-al_2013a} were inferred under the assumption that the candidates were $\sim$20\,kpc away. Therefore, metallicities that are higher than expected are consistent with the targets being notably closer, making their association with the Orphan Stream tenuous. It is also worth noting that these targets are the farthest candidates ($|B_{\rm Orphan}| \sim 0.5$\,$^\circ$) from the best-fit Orphan Stream orbital plane deduced by \citet{newberg;et-al_2010}.

The three high-probability targets have velocities within 2.4\,km s$^{-1}$ of each other, consistent with the Orphan Stream velocities, as well as the low velocity dispersion observed in the stream. The metallicities of these targets are also in reasonable agreement with those found from low-resolution spectroscopy, implying the candidates are $\sim$20\,kpc away at approximately the same distance as the Orphan Stream. 

Given a stream this distant, proper motions can only be utilised to positively exclude stream members, as high or statistically significant proper motions would suggest a nearby interloper. On the basis of the observable data we can deduce that the lower probability members, OSS 3 and OSS 18, are unlikely disrupted Orphan Stream members. For the remainder of this discussion, we consider OSS 6, 8 and 14, to be true disrupted members of the Orphan Stream satellite. These stars are identifiable by an encapsulated circle around every marker in all figures throughout this text.

\subsection{Stream Chemistry}

% Points:
% + Stream has a wide range in metallicity.
% + The stream has low [alpha/Fe] w.r.t to the halo of the same metallicity
%	-> low [alpha/Fe] characteristically observed in dwarf galaxies, and high [alpha/Fe]'s generally seen in globular clusters.
%	-> This should not be used as a litmus test.
%	-> Some dSph's have a large range in [alpha/Fe]'s, extending up to typical values seen in field stars.
%	-> Abundances indicate that SN Ia did not start to contribute until after [Fe/H] > -1.7 dex (in the Galaxy it occurred ~ -1).
%	-> Thus, [alpha/Fe] element abundances suggest a dwarf spheroidal progenitor.
% + The alpha-process occurs when neutron-rich, alpha-rich gas is out of nuclear statistical equilibrium, and has been identified as a possible site for light r-process element production.
%	-> If true, then if alpha-process contributes to [alpha/Fe] and light r-process in the Galaxy, but not in dSph's, lower [alpha/Fe] and [Y/Eu] expected.
%	-> We are only able to ascertain upper limits for [Y/Eu] in our stream stars, which show them to be lower than halo stars at the same metallicity. 
%	-> 
\subsubsection{Metallicity Distribution Function}
Although the sample size is small, the three Orphan Stream members have a wide range in metallicity, ranging from $-1.58$ to as metal-poor $-2.66$\,dex. From three members not much can be said about the metallicity distribution function (MDF), other than to say it is wide, and inconsistent with a mono-metallic population. Although we believe all 3 stars are disrupted Orphan Stream members, the likelihood of a halo interloper is non-zero, but unlikely to be more more than one candidate. Stars as metal-poor as OSS 14 (with [Fe/H] = $-2.66$) in the halo are somewhat uncommon. If we apply the color selection of \citet{casey;et-al_2013} to models of the Milky Way, we find the likelihood that OSS 14 -- our most metal-poor stream member -- is X with the Besan\c{c}on model and Y with the GALAXIA model \citep{robin;et-al_2013, sharma;et-al_2012}. Finding stars around [Fe/H] $\sim -1.7$ is more likely, but assuming OSS 14 is a true Orphan Stream member and we have a maximum of one halo interloper, we can report a wide metallicity range in the Orphan Stream. This is consistent with the low-resolution spectroscopic study of \citet{casey;et-al_2013a}, which has been recently confirmed with observations of BHB stars by \citet{sesar;et-al_2013}.

The Orphan Stream stars have low [$\alpha$/Fe] abundance ratios ($\sim$0.22) with respect to the Milky Way sample ($\sim$0.40\,dex). The two Orphan candidates that we deduced to be field interlopers also have low [$\alpha$/Fe] ratios than their Milky Way counterparts of a similar metallicity. However, these stars are consistent with the spread in [$\alpha$/Fe] abundances observed in the halo as a result of Type Ia supernovae beginning to significantly contribute to the halo chemistry near [Fe/H] $\sim$ -1 dex.

\subsubsection{$\alpha$-element abundance ratios}
In the dSph galaxies, Type Ia supernovae begin to show a significantly contribution to the [$\alpha$/Fe] abundances near [Fe/H] $> -1.7$\,dex,  with the exception of Draco which shows low [$\alpha$/Fe] at all metallicities. Some dwarf spheroidal stars also have a large range in [$\alpha$/Fe] values, increasing up to typical values seen in field stars. Additionally, the thick disk is known to comprise of both a high-$\alpha$ and low-$\alpha$ population. Thus, although low [$\alpha$/Fe] abundance ratios have characteristically been observed in dwarf galaxy stars, an [$\alpha$/Fe] abundance ratio should not be used as a litmus test for accretion. Given these stars are believed to be Orphan Stream members already, we can say that our [$\alpha$/Fe] abundance ratios are consistent with the star formation environment of a typical Milky Way dwarf galaxy. 


\subsubsection{[Ba/Y]}

\begin{figure}[h]
	\includegraphics[width=\columnwidth]{./Ba-Y.pdf}
	\caption{Ba-Y.}
	\label{fig:ba-y}
\end{figure}


% Peculiar chemical signatures
\subsection{Possible Parent Systems}
A number of associations have been proposed between the Orphan Stream and known Milky Way satellites. However, most have been shown to be either unlikely or implausible. Although Segue 1 has appeared as the most likely (known) satellite, a link with the globular cluster NGC 2419 has also been proposed \citet{bruens;et-al_2011}. Here we discuss the possibility of these associations in light of the data presented here.


\subsubsection{NGC 2419}
The width and length of a tidal tail are a clear indication to the nature of the parent satellite. For the case of the Orphan Stream, these characteristics favour a dwarf satellite. However, given the large apogalacticon of 90\,kpc for the Orphan Stream, a relationship has been proposed with the distant -- yet luminous -- globular cluster NGC 2419, also at $\sim90$\,kpc. Stars in globular clusters are known to exhibit peculiar chemical patterns, most notably in an anti-correlation between sodium and oxygen abundances. NGC 2419 is particularly special because it shows an unusual anti-correlation between magnesium and potassium. This signature is so far unique to NGC 2419, and illustrates the most extreme cases of Mg-depletion and K-enhancement seen anywhere in the Milky Way or its satellite systems.

% Ni and Na/Fe abundances?

We do not find any of our Orphan Stream stars to be extremely depleted in [Mg/Fe] or enhanced in [K/Fe], as found in NGC 2419 by \citet{coco;et-al}. If we follow the \citet{who} nomenclature and consider two `populations' in NGC 2419: a Mg-poor, K-rich and  a Mg-normal, K-poor sample, then a simple K-S test demonstrates that we can be X\% confident that the Orphan Stream members are not drawn from the Mg-poor NGC 2419 population. We cannot make such statements about the Mg-normal population, since they are typical for the Milky Way. The Mg-poor and Mg-normal sub-populations each comprise about 50\% of the total NGC 2419 sample. Given the uniqueness of the Mg-K anti-correlation, a single Orphan Stream star with significant Mg-depletion or K-enhancement would have provided an exciting hint for the Orphan Stream parent. In any case, the Orphan Stream's large metallicity spread indicates that the stream is unlikely to be associated with NGC 2419, given the cluster's small dispersion in overall metallicity. We therefore firmly rule out any association between NGC 2419 and the Orphan Stream.

\begin{figure}[h]
	\includegraphics[width=\columnwidth]{./Mg-K.pdf}
	\caption{Magnesium and potassium abundances for stars in the globular cluster NGC 2419 from \citet{cohen;et-al} ($\vartriangle$) and \citet{mucciarelli;et-al} ($\triangledown$). Program stars and Orphan Stream candidates are shown with the same markers used in Figure \ref{fig:alpha-fe}.}
	\label{fig:mg-k}
\end{figure}


\subsubsection{Segue 1}
The only known satellite system with an unclear association with the Orphan Stream is Segue 1. Segue 1 lies close to the perigalacton of the Orphan Stream. The original discovery by \citet{belokurov;et-al_2007a} suggested the system was an extended globular cluster, until the spectroscopic survey by \citet{simon;et-al_2011} showed the system was a dark-matter dominated dwarf galaxy. It's on-sky position lies a mere X degrees from the Orphan Stream plane, and ~Y degrees from the celestial equator cutoff of the Orphan Stream.

% velocity dispersion
Segue 1 has a very low velocity dispersion, less than 4\,km s$^{-1}$. In the closest part of the Orphan Stream, \citet{casey;et-al_2013} and \citet{newberg;et-al_2010} measured a similarly low velocity dispersion from different Orphan stream samples. The galactocentric velocity of the closest part of the stream (X\,$^\circ \Lambda_{\rm Orphan}$, $V_{\rm gsr} = $? \,km s$^{-1}$) matches Segue 1 almost identically ($V_{gsr} = ?$\,km s$^{-1}$). 

The characteristics of the stream are clues to the nature of its progenitor: the arc length and intrinsic width suggests the parent system must be dark matter dominated. Similarly, Segue 1 is the most dark matter-dominated system known. This fact alone has motivated several groups () to search for direct evidence of dark matter annihilation within Segue 1. Their observations place upper limits of Z, consistent with some of the most precise dark matter estimates from theory () and velocity dispersion observations (). The similarities between Segue 1 and the undiscovered parent of the Orphan Stream are striking. However, there is some doubt as to whether Segue 1 has experienced any tidal effects ().

% Metallicity gradient? Maybe after next paragraph

The chemistry of Segue 1 is particularly relevant. Segue 1 hosts an extremely wide range in stellar metallicities: from X to Y dex. \citet{simon;et-al_2011} and \citet{norris;et-al_2010} demonstrated the mean to be [Fe/H] = Z, with a spread of X. Some of the most metal-poor stars in Segue 1 extremely carbon enhanced ([C/Fe] > Z) and show no enhancement of neutron capture elements. From low-resolution spectroscopy, \citet{casey;et-al_2013a} found the Orphan Stream to have an intrinsic metallicity spread of $\sigma([$Fe/H]$) = 0.56$\,dex, with candidates ranging from [Fe/H] = X to Y dex. This level of carbon enhancement is reminiscent of the Milky Way halo, further suggesting the dwarf spheroidal galaxies are the building blocks of the Galaxy. In our three Orphan Stream members we confirm a wide range in metallicities, but not to the same extent observed by \citet{norris;et-al_2010} and \citet{simon;et-al_2011} in Segue 1. Our most metal-poor member has [Fe/H] = -2.66 and [C/Fe] = X. 

% Does the fact that CEMP-no stars exist suggest the more metal-rich stars should not be particularly over-abundant in neutron-capture elements?


% on sky position?
% ships in the night
% velocity dispersion


% Extent of metallicities?
% [C/Fe]
% [alpha/Fe]
% [Ba/Y], [Eu/Y]


% [Ba/Y] offset in dSph and halo stars seen by Shetrone (2003) and Venn (2004)
% large mass stars responsible for production of Sr, Y, Zr. Truncating the IMF at the high end would reduce these abundances and reduce the overall [alpha/Fe] because higher mass stars are expected to produce higher [alpha/Fe]. Explained in Shetrone (2003). Used by Smith (2002) for LMC stars.





\begin{figure}[h]
	\includegraphics[width=\columnwidth]{./heavy-Fe.pdf}
	\caption{$s$-process abundance ratios for standard halo and Orphan Stream candidates. Upper limits are indicated where no measurements were available.}
	\label{fig:heavy-fe}
\end{figure}



% Extent of metallicities?
% [C/Fe]
% [alpha/Fe]
% [Ba/Y], [Eu/Y]

% Plausible associations:
% - NGC 2419
% - Segue 1

\section{Conclusions}
\label{sec:conclusions}

We present a chemical analysis of five Orphan stream candidates, three of which we confirm are true disrupted members from the Orphan stream parent satellite. The two non-members were both originally identified to have `low' or `medium' probability of membership from low-resolution spectroscopy. The three disrupted members show chemical distinctiveness from the halo at the same metallicity, suggesting the stars were accreted. A large metallicity spread is present, further implying the Orphan Stream parent is not mono-metallic. The spread in overall metallicity is representative of the stochastic chemical enrichment observed in the Milky Way dwarf spheroidal galaxies. Detailed chemical abundances confirm this proposition. On the basis of chemistry, we exclude the extended globular cluster NGC 2419 as a plausible parent to the Orphan Stream, and conclude that Segue 1 remains the most likely {\it known} satellite. It is entirely possible that the parent satellite has not been found. Deep, wide multi-band photometry surrounding the edge of the Orphan Stream near the celestial equator would resolve this outstanding problem. Furthermore, we postulate that follow-up high-resolution spectroscopy of the remaining six high probability members would further demonstrate the Orphan stream's chemical distinctiveness from the halo.


\acknowledgements
A.R.C. acknowledges the financial support through the Australian Research Council Laureate Fellowship LF0992131. SK and GDaC acknowledge the financial support from the Australian Research Council through Discovery Programs DP0878137 and DP120101237.


\bibliographystyle{apj}
\bibliography{bibliography}
\end{document}
