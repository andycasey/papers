%!TEX TS-program = pdflatex
\documentclass{emulateapj}

\usepackage{verbatim} % for comment blocks
\usepackage{pdflscape} % landscape tables

\shorttitle{Chemistry of the Orphan Stream II}
\shortauthors{Casey et al}

\begin{document}

\title{Chemistry of the Orphan Stream II: The First High-Resolution Spectroscopic Analysis on the Orphan Stream}

\author{Andrew R. Casey\altaffilmark{1,2}, Stefan C. Keller\altaffilmark{1}, Frebel, Anna\altaffilmark{2}, Gary Da Costa\altaffilmark{1}, Elizabeth Maunder\altaffilmark{1}}
\altaffiltext{1}{Research School of Astronomy \& Astrophysics, Australian National University, Mount Stromlo Observatory, via Cotter Rd, Weston, ACT 2611, Australia; \email{arcasey@mso.anu.edu.au}}
\altaffiltext{2}{Massachusetts Institute of Technology, Kavli Institute for Astrophysics and Space Research,
77 Massachusetts Avenue, Cambridge, MA 02139, USA}


\begin{abstract}
test
\end{abstract}

\keywords{Galaxy: halo, structure --- Individual: Orphan Stream --- Stars: K-giants}

\section{Introduction}

% galactic structure, accretion, build up of galaxies
In the last decade the importance of hierarchical merging for the formation of galaxies has become unambiguously clear. In $\Lambda$CDM cosmology, small clusters conglomerate together to form dwarf galaxies. If these dwarf galaxies become sufficiently close to larger galaxies, they begin infalling and tidal forces cause a stream of stars to be left in their wake. These disrupted stars accrete onto the Milky Way, contributing to a significant fraction of the halo \citet{Bell;et-al_2008}.

 
% how streams form
% streams agree with lambda cdm
% streams contribute a significant part to the halo

% dominant streams have been identified by their kinematic signatures 
  The Milky Way is the best laboratory to investigate the extent accretion plays in galaxy formation and evolution. Historically, the signatures of accretion events have been identified either through prominent stellar over-densities [citations needed] or by co-moving groups; numerous spatially coincident stars moving with the same velocity [citations needed]. The kinematics of stars in streams allow us to reconstruct the orbital history of the host and place constraints on the Milky Way potential, including the shape and extent of dark matter in the Galaxy [citation needed]. 
 
  
% wealth of data provided by SDSS, allowed orphan stream to be discovered independently
With increasing all-sky survey data being made openly available, searching for stellar over-densities has become extremely efficient [cite koposov]. Using simple color cuts in the SDSS DR5 [citation needed] data set, both \citet{Grillmair_2006} and \citet{Belokurov;et-al_2007} independently discovered the Orphan Stream as a coherent structure spanning over $60^\circ$ in the sky. The Orphan Stream is uniquely distinct from  other stellar streams uncovered in the halo. It has an extremely low surface brightness and is only $\approx2^\circ$ wide \--- which is significantly narrower than other structures found in the Field of Streams \citep{Belokurov;et-al_2006} like Sagittarius ($15^\circ$ wide; [citation needed]). Moreover, the Orphan Stream was aptly named by \citet{Belokurov;et-al_2007} because unlike most other streams, no associated host was found in the SDSS data set, and no existing known system has been reliably associated. The Orphan stream extends well past the celestial equator, outside of the SDSS footprint. Unfortunately the stream has a sufficiently low surface density \citet{Newberg;et-al_2010} found prevents it from being detected within the existing SuperCOSMOS survey \citep{Hambly;et-al_2001}. It is unlikely that this region will be adequately imaged until the SkyMapper telescope commences the Southern Sky Survey \citep{Keller;et-al_2007}. At least until then, it remains a orphaned stream without a parent system.
  
  % Should I discuss the observational bias, e.g. that we need the smaller ones to really start to pinpoint down the MW potential?

% orphan stream is aptly named and is unique to all other streams. the parent remains lost, although a lot of work has been put in to associating the stream
In the absence of observational data to identify an undiscovered parent host, there has been much effort on associating the Orphan Stream with an existing Milky Way satellite [citations needed]. Given the XX and XX, it is likely the parent host is a dwarf spheroidal galaxy [citations needed], although \citet{Grillmair_2006} predicts the host to be completely disrupted. In the search for an associated system, \citet{Belokurov;et-al_2007} first demonstrated that the Complex A, Ruprecht 106, and Palomar 1 all lie on the same great circle path of the Orphan stream, speculating some association between the four. 

% Let's deal with the Complex A controversy first
A detailed analysis on the Orphan Stream orbit is required to examine any potential associations. Kinematics and/or proper motions are required to adequately determine an orbit.
\citet{Belokurov;et-al_2007} listed admittedly two tenuous kinematic observations along the stream, one of which it later appears was erroneous. 

% talk about fellhauer

Using these measurements \--- the best, and only at the time \--- and assuming Complex A and the Orphan Stream shared the same orbit, \citet{Jin;Lynden-Bell_2007} found a representative orbit to match observations. Unfortunately their predicted heliocentric velocities that were inconsistent with \citet{Fellhaur;et-al_2007} and later observations by \citet{Newberg;et-al_2010}. Moreover, distances estimated by \citet{Jin;Lynden-Bell_2007} were under-predicted by a factor of three.



In contrast to these simulations, \citet{Sales;et-al_2007} using a X found an orbit which reproduced the velocities, positions and characteristics of the stream, but found no evidence of association with Complex A, or UMaII. 
Their findings indicate that the parent system is likely to be dark matter dominated, and have similar characteristics to some of the Milky Way dwarfs already known (XX, XX and XX). Only self-gravitating systems dominated by dark matter can survive sufficiently long enough to form a stream stretching $\approx55^\circ$ across the sky. 

% talk more about Sales

As the SEGUE [acronym needed] data became available, \citet{Newberg;et-al_2010} employed a detailed selection method in order to identify BHB stars in the SDSS data sample.
% blah blah blah

% perhaps mention Sales et al explaining how low surface brightness streams cant be found without spectroscopic surveys

More recently, \citet{Keller;et-al_2012} targeted FGK giants on the Southern end of the SDSS footprint. As the stream crosses the celestial equator it is closest to us and the surface brightness peaks, making it an ideal place to identify stream members. Using inclusive SDSS cuts they targeted hundreds of stars with medium resolution spectroscopy. By identifying likely stream giants based on kinematics and surface gravity estimates, they identified X likely Orphan Stream giants. In this paper we present a detailed chemical analysis on five of those members using high resolution spectroscopic data from the Magellan Clay telescope. Details regarding the observations are outlined in Section \ref{sec:observations}, followed by X.

\section{Observations}
We present high resolution follow-up observations using the Magellan Inamori Kyocera Echelle (MIKE) spectrograph.

% details of the resolution that MIKE provides, etc

\subsection{Target Selection}
% this is double-up
The Orphan Stream has an extremely low surface brightness [citations needed] of XX, and there is a well-described distance gradient along the Orphan stream \citep{Belokurov;et-al_2007,Newberg;et-al_2010}. Therefore the best place to identify stream members within the SDSS sky coverage is near the celestial equator where the stream is closest to us, right at the edge of the SDSS footprint.


\subsection{High Resolution Observations}


\begin{deluxetable*}{lccccccc}
\tablecolumns{2}
\tabletypesize{\scriptsize}
\tablecaption{Observed Targets\label{tab:observed-targets}}
\tablehead{
	\colhead{Star} &
	\colhead{$\alpha$} &
	\colhead{$\delta$} &
	\colhead{Air mass} &
	\colhead{S/N\tablenotemark{a}} &
	\colhead{$V_{\mbox{helio}}$} &
	\colhead{Comment} \\
 & (J2000) & (J2000) & & (px$^{-1}$) & (km s$^{-1}$) &
}
\startdata
HD 136316 & 15 22 17.2 & $-$53 14 13.9 & 1.118 & 335 & $-38.2 \pm 1.1$       & \\
HD 141531 & 15 49 16.9 &\phs09 36 42.5 & 1.309 & 280 & $\phn\phn2.6 \pm 1.0$ & \\
HD 142948 & 16 00 01.6 & $-$53 51 04.1 & 1.107 & 271 & $\phn30.3 \pm 0.9$    & \\
HD 41667  & 06 05 03.7 & $-$32 59 36.8 & 1.005 & 272 & $297.8 \pm 1.7$       & \\
HD 44007  & 06 18 48.6 & $-$14 50 44.2 & 1.033 & 239 & $163.4 \pm 1.3$       & \\
HD 47536  & 06 37 47.7 & $-$32 20 20.1 & 1.002 & 257 & $\phn78.9 \pm 1.2$    & \\
HD 76932  & 08 58 44.2 & $-$16 07 54.2 & 1.158 & 289 & $119.2 \pm 1.2$       & \\
HD 84903  & 09 47 19.3 & $-$41 27 04.9 & 1.260 & 294 & $\phn77.7 \pm 1.4$    & \\
HD 59984  & 07 32 05.7 & $-$08 52 56.1 & 1.111 & 402 & $\phn55.7 \pm 0.5$    & \\
HD 60060  & 07 29 59.6 & $-$52 39 04.3 & 1.127 & 414 & $\phn25.6 \pm 1.1$    & \\
HD 60228  & 07 30 30.8 & $-$54 23 58.6 & 1.139 & 338 & $\phn48.6 \pm 1.7$    & \\
OSS 1     & 10 46 50.6 & $-$00 13 17.9 & 1.363 &  48 & $218.2 \pm 1.7$       & \\
OSS 2     & 10 47 17.8 &\phs00 25 06.9 & 1.995 &  59 & $222.2 \pm 1.3$       & \\
OSS 3     & 10 47 30.3 & $-$00 01 22.6 & 1.156 &  49 & $226.4 \pm 1.3$       & \\
OSS 4     & 10 49 08.3 &\phs00 01 59.3 & 1.881 &  48 & $227.5 \pm 2.1$       & Poor seeing\\
OSS 5     & 10 50 33.7 &\phs00 12 18.3 & 1.295 &  31 & $249.0 \pm 2.6$       & Poor seeing \\
HR 6141   & 16 30 12.3 & $-$25 06 52.0 & 1.003 & 406 & \nodata               & Telluric
\enddata
\tablenotetext{a}{S/N measured at 6000 \AA{} for each target.}
\end{deluxetable*}

% HDs 59984, 76932, 136316, 84903, 44007, 142948
% OS 2,4,5,6,8 (file name convention) here is OS 1,2,3,4,5


\subsection{Data Reduction}

The data were reduced using the latest version of the MIKE pipeline outlined in \citet{Kelson;2003}. For comparison we also reduced our HD 59984 spectra using standard methods in \textsc{IRAF} and found no noteworthy difference between the reduced products. Individual frames for each object were co-added before analysis. We carefully normalised every echelle order using cubic splines. Overlapping normalised orders on each CCD were stitched together, such that a continuous normalised spectra from each CCD results. No telluric corrections were made, as they do not affect the spectral regions we are interested in.

\subsection{Radial Velocities}

We measured radial velocities for every target by cross-correlating each normalised red  spectrum with a synthetic template. Our synthetic template was generated in \textsc{MOOG} using an $\alpha$ element enhanced \citet{Castelli-Kurucz;2004} model atmospheres with $T_{eff} = 4500$ K, $\log{g} = 2.5$, [Fe/H] = -1.5 and no convective overshoot. The line list of \citet{Kirby;et-al_2008} was employed to generate spectra between $845$ nm $< \Lambda < 870$ nm. Both the blue and red normalised frames were shifted to rest before measuring equivalent widths. A number of radial velocity and halo standards were observed during this program, and our heliocentric velocities agree excellently with literature values (Table \ref{tab:observed-targets}).

\subsection{Line List}
\label{sec:line-list}


% table

\subsection{Line Measurements}
An automatic routine has been developed to determine the local continuum and robustly measure the equivalent width of an absorption line. First, we estimate the local continuum surrounding the rest wavelength (Table \ref{tab:line-list}) by iteratively fitting a low order polynomial to $10$ \AA{} either side of the rest wavelength, and clipping only extreme outliers. This process does not determine the final local continuum; all we need is an initial estimate.

Given a rest wavelength and our local continuum estimate, a Gaussian absorption profile is fitted in an iterative process. During this process we have three free parameters: the rest wavelength, the FWHM of the Gaussian profile, and the flux level at the profile trough. The initial value for the minimum flux is taken as the flux level at rest wavelength, and 0.1 \AA{} is used as the initial FWHM for the Gaussian profile. The actual FWHM of our final profile and the subsequent equivalent width measured is largely independent of the initial FWHM provided; only an initial guess is required to decrease computational cost. The difference between our calculated profile and the actual flux is calculated at each iteration, and we use least-squares minimisation to arrive at the best fitting line profile. 

Once this process is complete, we use the measured line FWHM in order to decide where the true continuum surrounding the line should start. Again, we iteratively fit a low order polynomial to the surrounding $10$ \AA{} \--- excluding $3\sigma$ of our measured profile\footnote{Recall that FWHM $\approx 2.355\sigma$} \--- but in this step the outlier clipping is more involved: after each iteration we identify flux points as outliers if $>2.5-\sigma$, and we identify sequentially occurring outlier points as probable minima of an absorption line. At each group of outlier points we attempt to fit a Gaussian absorption profile in the same process described above. If a profile is successfully fitted, then we exclude all points within $3-\sigma$ of the detected line from any more continuum determination. In this process, absorption lines are automatically detected and excluded, leaving only the continuum.

When multiple lines are measured from our line list there are stringent quality constraints put in place to eliminate false positive measurements: the final rest wavelength must be within <$10$ \AA{} to the initial guess, the FWHM of each line should be reasonable with respect to all other lines, and the $\chi^2$ difference between the final profile and the spectrum must be reasonable. The $\chi^2$ constraint helps identify measurements where nearby lines are blended. 

For spectra which has been perfectly normalised and has high S/N, the method described here will always find the continuum to be unity. For low S/N objects, or where the normalisation is sub-par, our technique robustly determines the true local continuum. We must stress that this process is not 'blind'; the profile and local continuum determination for every measured line has been carefully inspected by the authors. All false positive measurements were discarded during the inspection process, and no lines required manual re-measurement.

%We have verified the accuracy of our automatically determined equivalent widths in two ways:
We have verified the accuracy of our automatically determined equivalent widths both internally and externally. For an internal comparison we have compared careful hand measurements of the equivalent widths against those determined automatically for two stars: the bright metal-rich dwarf HD 59984, and our metal-poor giant target OSS 4. The results of this comparison are shown in Figures \ref{fig:hd59984-ew-comparison} and \ref{fig:oss4-ew-comparison}, and demonstrate our measurement code at varying S/N. HD59984 has a S/N $\approx XX$ px$^{-1}$ at 600 nm, and the difference between equivalent width measurement methods is negligible. Our target star OSS 4 has a much more modest S/N of $\approx50$ px$^{-1}$ at 600 nm, and the two techniques still agree excellently.

These internal comparisons validate the effectiveness of the local continuum fitting within a wide bracket of S/N ratios. To verify our equivalent width accuracy to an external source, we have automatically measured equivalent widths for the metal-poor sub-giant HD140283 and compared our measurements to those of \citet{Norris;et-al_1996}. 

The comparison is shown in Figure \ref{fig:norris-ew-comparison}, and the two techniques agree excellently. There is a systematic offset of XX m\AA{} and random scatter of YY m\AA{} between the two techniques. We are confident that our automatic measurement technique can reliably determine equivalent width measurements that are consistent with the literature, over a wide range of evolutionary phases, metallicities and signal quality.




\newpage

\begin{deluxetable*}{lcccccccccl}
\tablecolumns{1}
\tabletypesize{\scriptsize}
\tablecaption{Stellar Parameters\label{tab:stellar-parameters}}
\tablehead{
& \multicolumn{4}{c}{This work}  &
& \multicolumn{4}{c}{Literature} & \\
  \cline{2-5} \cline{7-10} \\
  \colhead{Star} &
  \colhead{$T_{\mbox{eff}}$} &
 \colhead{$\log{(g)}$} &
	\colhead{[Fe/H]} &
	\colhead{$v_t$} 
	&&
	\colhead{$T_{\mbox{eff}}$} &
	\colhead{$\log{(g)}$} &
	\colhead{[Fe/H]} &
	\colhead{$v_t$} &
	\colhead{Reference} \\
 & [K] & [dex] & [dex] & [km s$^{-1}$] 
&& [K] & [dex] & [dex] & [km s$^{-1}$] & 
}
\startdata
HD 136316 & $4370$ & $1.00$ & $-1.99$ & $1.95$ 
         && $4469$ & $1.29$ & $-1.74$ & $1.90$ & \citet{Gratton;et-al_2000} \\
HD 141531 & $4330$ & $0.90$ & $-1.74$ & $1.80$ 
         && $4335$ & $1.11$ & $-1.62$ & $1.50$ & \citet{Gratton;et-al_2000} \\
HD 142948 & $4870$ & $2.20$ & $-0.78$ & $1.45$ 
         && $4713$ & $2.17$ & $-0.77$ & $1.38$ & \citet{Gratton;et-al_2000} \\
HD 41667  & $4600$ & $1.70$ & $-1.23$ & $1.50$  
         && $4605$ & $1.88$ & $-1.16$ & $1.44$ & \citet{Gratton;et-al_2000} \\
HD 44007  & $4800$ & $1.80$ & $-1.77$ & $1.55$ 
         && $4850$ & $2.00$ & $-1.71$ & $2.20$ & \citet{Fulbright_2000} \\
HD 47536  & & & & 
         &&\nodata &\nodata & \nodata &\nodata & \\
HD 76932  & & & & 
         && $5900$ & $4.20$ & $-0.70$ & $1.25$ & \citet{Fulbright_2000} \\
HD 84903  & & & & 
         && $4500$ & $0.80$ & $-2.60$ & $2.00$ & \citet{Francois_1996} \\
HD 59984  & & & & 
         && & & & & Binary system.\\
HD 60060  & & & & 
         &&\nodata &\nodata & \nodata &\nodata & \\
HD 60228  & & & & 
         &&\nodata &\nodata & \nodata &\nodata & \\
OSS 1     & 5040 & 2.80 & -1.22 & 1.40 && & & & & \\
OSS 2     & 4550 & 0.80 & -1.78 & 1.85 && & & & & \\
OSS 3     & 4780 & 1.55 & -1.79 & 1.70 && & & & & \\
OSS 4     & 4580 & 0.80 & -2.75 & 2.05 && & & & & \\
OSS 5     & 5250 & 3.00 & -0.80 & 1.85 && & & & & 
\enddata
\end{deluxetable*}


\newpage
\newpage

\newpage
\newpage

\begin{deluxetable*}{lccccccccccccccccc}
\tablecolumns{18}
\tablewidth{\textwidth}
\tabletypesize{\scriptsize}
\tablecaption{Standard Star Chemistries\label{tab:stellar-parameters}}
\tablehead{
&  \multicolumn{5}{c}{HD 142948} & \colhead{} &
	\multicolumn{5}{c}{HD 141531} & \colhead{} &
	\multicolumn{5}{c}{HD 44007} \\
	\cline{2-6} \cline{8-12} \cline{14-18} \\
\colhead{Species} 
 & $\log\epsilon$(X) & [X/H] & [X/Fe] & $N$ & $\sigma$ &
 & $\log\epsilon$(X) & [X/H] & [X/Fe] & $N$ & $\sigma$ &
 & $\log\epsilon$(X) & [X/H] & [X/Fe] & $N$ & $\sigma$
}
\startdata

Na \textsc{I} & 5.99 & $-$0.25 & \phn0.53 &  1 & \nodata\tablenotemark{a} % HD 142948
             && 4.56 & $-$1.68 & \phn0.04 &  2 & 0.07    % HD 141531
             && 4.53 & $-$1.71 & \phn0.06 &  3 & 0.05 \\ % HD 44007
Mg \textsc{I} & 7.21 & $-$0.39 & \phn0.39 &  4 & 0.06
             && 6.26 & $-$1.34 & \phn0.38 &  4 & 0.12
             && 6.16 & $-$1.44 & \phn0.33 &  3 & 0.04 \\
O \textsc{I}  & 8.54 & $-$0.15 & \phn0.63 &  2 & 0.05
             && 7.51 & $-$1.18 & \phn0.54 &  2 & \phn0.00\tablenotemark{a}
             && 7.50 & $-$1.19 & \phn0.58 &  1 & \nodata\tablenotemark{a} \\
Si \textsc{I} & 6.67 & $-$0.84 &  $-$0.06 &  2 & 0.07
             && 5.86 & $-$1.65 & \phn0.07 &  3 & 0.20
             && 6.19 & $-$1.32 & \phn0.45 &  3 & \phn0.02\tablenotemark{a} \\
Ca \textsc{I} & 5.83 & $-$0.51 & \phn0.27 &  8 & 0.07
             && 4.91 & $-$1.43 & \phn0.29 & 12 & 0.09
             && 5.02 & $-$1.32 & \phn0.45 & 16 & 0.08 \\
Ti \textsc{I} & 4.26 & $-$0.69 & \phn0.09 & 15 & 0.11
             && 3.09 & $-$1.86 &  $-$0.14 & 19 & 0.11
             && 3.32 & $-$1.63 & \phn0.14 & 19 & 0.08 \\
Ti \textsc{II}& 4.53 & $-$0.42 & \phn0.36 & 18 & 0.11
             && 3.53 & $-$1.42 & \phn0.30 & 22 & 0.11
             && 3.50 & $-$1.45 & \phn0.32 & 32 & 0.12 \\        
Sc \textsc{II}& ?    &   ?     &    ?     &  ? &  ?
             && 1.60 & $-$1.55 & \phn0.17 &  7 & \phn0.04\tablenotemark{a}
             && 1.46 & $-$1.69 & \phn0.08 &  9 & 0.06 \\
Fe \textsc{I} & 6.72 & $-$0.78 & \phn0.00 & 104& 0.14
             && 5.78 & $-$1.72 & \phn0.00 & 100& 0.11
             && 5.72 & $-$1.78 & \phn0.00 & 135& 0.10 \\
Fe \textsc{II}& 6.73 & $-$0.77 & \phn0.01 & 13 & 0.18
             && 5.86 & $-$1.64 & \phn0.08 & 11 & 0.10
             && 5.70 & $-$1.80 &  $-$0.02 & 14 & 0.11 \\
V \textsc{I}  & 6.27 &\phn2.34 & \phn3.12 &  2 & \phn0.02\tablenotemark{a}
             && 4.69 &\phn0.76 & \phn2.50 &  2 & 0.22
             && 4.63 &\phn0.70 & \phn2.47 &  3 & \phn0.04\tablenotemark{a} \\
Cr \textsc{I} & 4.66 & $-$0.98 &  $-$0.20 &  6 & 0.07
             && 3.61 & $-$2.03 &  $-$0.29 & 10 & 0.08
             && 3.67 & $-$1.97 &  $-$0.20 & 13 & 0.06 \\
Cr \textsc{II}& 4.83 & $-$0.81 &  $-$0.03 &  3 & \phn0.04\tablenotemark{a}
             && 4.06 & $-$1.58 & \phn0.16 &  2 & 0.07
             && 4.00 & $-$1.64 & \phn0.13 &  5 & 0.05 \\
Mn \textsc{I} & 4.98 & $-$0.45 & \phn0.33 &  4 & 0.12
             && 3.72 & $-$1.71 & \phn0.03 &  5 & 0.18
             && 3.44 & $-$1.99 &  $-$0.22 &  6 & 0.22 \\
Co \textsc{I} & 4.06 & $-$0.93 &  $-$0.15 &  3 & \phn0.01\tablenotemark{a}
             && 2.97 & $-$2.02 &  $-$0.28 &  3 & 0.15
             && 3.13 & $-$1.86 &  $-$0.09 &  4 & 0.10 \\
Ni \textsc{I} & 5.52 & $-$0.70 & \phn0.08 &  8 & 0.08
             && 4.33 & $-$1.89 &  $-$0.15 & 13 & 0.09
             && 4.34 & $-$1.88 &  $-$0.11 & 13 & 0.09 \\
\enddata
\tablenotetext{a}{Minimum uncertainty of $\pm$0.05 dex adopted for this measurement.}
\end{deluxetable*}

\newpage
\newpage


\begin{deluxetable*}{lccccccccccccccccc}
\tablecolumns{18}
\tablewidth{\textwidth}
\tabletypesize{\scriptsize}
\tablecaption{Orphan Stream Chemistry\label{tab:stellar-parameters}}
\tablehead{
&  \multicolumn{5}{c}{OSS 1} & \colhead{} &
	\multicolumn{5}{c}{OSS 2} & \colhead{} &
	\multicolumn{5}{c}{OSS 3} \\
	\cline{2-6} \cline{8-12} \cline{14-18} \\
\colhead{Species} 
 & $\log\epsilon$(X) & [X/H] & [X/Fe] & $N$ & $\sigma$ &
 & $\log\epsilon$(X) & [X/H] & [X/Fe] & $N$ & $\sigma$ &
 & $\log\epsilon$(X) & [X/H] & [X/Fe] & $N$ & $\sigma$
}
\startdata
Na \textsc{I} & ? & ? & ? & ? & ?                              % OSS 1
             && $4.21$ & $-2.03$ &    $-0.23$ &   $3$ & $0.05$ % OSS 2
             && $4.42$ & $-1.82$ &    $-0.03$ &   $3$ & $0.07$ % OSS 3
             \\
Mg \textsc{I} & $6.56$ & $-1.04$ & \phs$0.16$ &   $4$ & $0.10$  
             && $5.90$ & $-1.70$ & \phs$0.10$ &   $6$ & $0.10$   
             && $5.95$ & $-1.65$ & \phs$0.15$ &   $5$ & $0.05$  \\
Si \textsc{I} & $6.57$ & $-0.94$ & \phs$0.26$ &   $4$ & $0.23$
             && $5.66$ & $-1.85$ &    $-0.05$ &   $3$ & $0.26$   
             && $6.20$ & $-1.31$ & \phs$0.43$ &   $3$ & $0.16$  \\
Ca \textsc{I} & $5.48$ & $-0.86$ & \phs$0.34$ &  $14$ & $0.14$
             && $4.79$ & $-1.55$ & \phs$0.25$ &  $17$ & $0.12$   
             && $4.87$ & $-1.47$ & \phs$0.32$ &  $18$ & $0.09$  \\
Ti \textsc{I} & $3.73$ & $-1.22$ &    $-0.02$ &  $21$ & $0.15$  
             && $3.00$ & $-1.95$ &    $-0.15$ &  $20$ & $0.16$   
             && $3.13$ & $-1.82$ &    $-0.03$ &  $23$ & $0.09$  \\
Ti \textsc{II}& $3.91$ & $-1.04$ & \phs$0.16$ &  $25$ & $0.10$  
             && $3.19$ & $-1.76$ & \phs$0.04$ &  $23$ & $0.14$   
             && $3.36$ & $-1.59$ & \phs$0.20$ &  $24$ & $0.09$  \\
Fe \textsc{I} & $6.24$ & $-1.26$ &    \nodata & $149$ & $0.16$  
             && $5.71$ & $-1.79$ &    \nodata & $112$ & $0.16$  
             && $5.71$ & $-1.79$ &    \nodata & $154$ & $0.12$  \\
Fe \textsc{II}& $6.19$ & $-1.21$ & \phs$0.05$ &  $15$ & $0.16$
             && $5.75$ & $-1.75$ &    $-0.04$ &   $8$ & $0.10$   
             && $5.71$ & $-1.79$ & \phs$0.00$ &  $20$ & $0.16$  \\
Sc \textsc{II}&&&&&  % OSS 1
             && $1.10$ & $-2.05$ &    $-0.27$ &   $7$ & $0.07$ 
             && $1.25$ & $-1.90$ &    $-0.11$ &  $10$ & $0.10$ \\
V \textsc{I}  &&&&&  % OSS 1
             && $4.50$ & \phs$0.57$ & $-1.21$ &   $4$ & $0.17$ 
             && $4.57$ & \phs$0.64$ & $-1.15$ &   $3$ & $0.03$ \\
Cr \textsc{I} &&&&&  % OSS 1
             && $3.63$ & $-2.01$ &    $-0.23$ &  $12$ & $0.08$ 
             && $3.63$ & $-2.01$ &    $-0.22$ &  $15$ & $0.11$ \\
Cr \textsc{II}&&&&&  % OSS 1
             && $3.81$ & $-1.83$ &    $-0.05$ &   $3$ & $0.11$ 
             && $3.99$ & $-1.65$ & \phs$0.14$ &   $3$ & $0.04$ \\
Mn \textsc{I} &&&&&  % OSS 1
             && $3.42$ & $-2.01$ &    $-0.23$ &   $4$ & $0.04$ 
             && $3.41$ & $-2.02$ &    $-0.23$ &   $6$ & $0.14$ \\
Co \textsc{I} &&&&&  % OSS 1
             && $3.14$ & $-1.85$ &    $-0.07$ &   $3$ & $0.19$ 
             && $3.00$ & $-1.99$ &    $-0.20$ &   $5$ & $0.17$ \\
Ni \textsc{I} &&&&&  % OSS 1
             && $4.24$ & $-1.98$ &    $-0.20$ &   $7$ & $0.07$ 
             && $4.32$ & $-1.90$ &    $-0.11$ &  $13$ & $0.12$ \\
             
\\ & \multicolumn{5}{c}{OSS 4} && \multicolumn{5}{c}{OSS 5} \\
     \cline{2-6} \cline{8-12} \\
Na \textsc{I} & $3.69$ & $-2.55$ & \phs$0.20$ &  $2$ & $0.04$
             && $5.17$ & $-1.07$ &    $-0.27$ &  $4$ & $0.13$ \\ 
Mg \textsc{I} & $5.11$ & $-2.49$ & \phs$0.26$ &  $7$ & $0.13$
             && $6.70$ & $-0.90$ &    $-0.10$ &  $5$ & $0.15$ \\
Si \textsc{I} & $5.38$ & $-2.13$ & \phs$0.62$ &  $3$ & $0.03$
             && $6.78$ & $-0.73$ & \phs$0.07$ &  $4$ & $0.07$ \\
Ca \textsc{I} & $3.75$ & $-2.59$ & \phs$0.16$ &  $9$ & $0.10$
             && $5.77$ & $-0.57$ & \phs$0.23$ & $10$ & $0.12$ \\
Ti \textsc{I} & $2.28$ & $-2.67$ & \phs$0.08$ & $13$ & $0.15$
             && $4.26$ & $-0.69$ & \phs$0.11$ & $15$ & $0.36$ \\
Ti \textsc{II}& $2.31$ & $-2.64$ & \phs$0.11$ & $24$ & $0.15$
             && $4.28$ & $-0.67$ & \phs$0.13$ & $13$ & $0.19$ \\
Fe \textsc{I} & $4.74$ & $-2.76$ &    \nodata & $89$ & $0.11$
             && $6.68$ & $-0.82$ &    \nodata & $99$ & $0.18$ \\
Fe \textsc{II}& $4.77$ & $-2.73$ & \phs$0.03$ & $13$ & $0.12$
             && $6.71$ & $-0.79$ &    $-0.03$ & $14$ & $0.16$ \\
\enddata
\end{deluxetable*}

\newpage
\newpage

\newpage
\newpage




\bibliographystyle{apj}
\bibliography{bibliography}
\end{document}
