We present two sets of spectroscopic observations: medium resolution spectroscopy with AAOmega on the Australian Astronomical Telescope, and the follow-up of Orphan stream candidates with high resolution echelle spectroscopy using MIKE on the Magellan Clay telescope. 

\subsection{Medium Resolution Observations}

\subsubsection{Target Selection}
Given the well-described distance gradient along the Orphan stream \citep{Belokurov;et-al_2007,Newberg;et-al_2010}, we have chosen to target stars closest to us, near the celestial equator and on the edge of the SDSS footprint.

% sdss cuts employed

\subsubsection{Observations}
AAOmega is a dual beam multi-object spectrograph which is capable of simultaneously observing spectra of 400 targets across a $2^\circ$ field of view. We employed the 1000I grating in the red arm which provides spectral coverage between $X \leq \lambda \leq Y$ nm at $R$ = XXXX, and the 580V grating in the blue arm that covers $X \leq \lambda \leq Y$ nm with a lower resolving power of $R \approx 1300$. The data were taken in visitor mode during Y 2009. 

The data were reduced using the supplied \textsc{2DFDR} reduction pipeline, which includes flat fielding, sky subtraction, and wavelength calibration. The median S/N obtained in the red arm for our fields is modest at 35 px$^{-1}$. However, with the presence of strong Ca \textsc{II} triplet lines we are able to ascertain reliable radial velocity measurements and reasonable estimates on metallicity \citep[][and references therein]{Starkenburg;et-al_2010}. 

\subsubsection{Data Analysis}

Radial velocities were measured by cross-correlating our normalised reduced spectra with a synthetic template of K-giant with a temperature of $4500$ K, $\log{g}$ = 2 and [Fe/H] = $-1.5$ across the range $845 \leq \lambda \leq 870$ nm. The equivalent widths for the Ca \textsc{II} lines were measured by E. M. and used to estimate metallicities using the \citet{Battaglia;et-al_2008} relationship and an assumed distance of $X$ kpc [citation needed]. In order to eliminate K dwarves from our sample we have measured the equivalent widths of the gravity-sensitive Mg \textsc{I}b lines at $518$ nm and employed the dwarf discriminant described in \citet{Casey;et-al_2012}.

% using a combination of cuts we have identified X likely K-giant candidates for high resolution spectroscopic follow-up. these candidates are highlighted in Figure X?

