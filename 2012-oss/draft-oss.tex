%!TEX TS-program = pdflatex
\documentclass{emulateapj}

\shorttitle{OSS on MIKE}
\shortauthors{Casey et al}

\begin{document}

\title{High Resolution Spectroscopic Analysis of Orphan Stream Giants}


\author{Andrew R. Casey\altaffilmark{1,2}, Stefan C. Keller\altaffilmark{1}, Frebel, Anna\altaffilmark{2}, Gary Da Costa\altaffilmark{1}, Elizabeth Maunder\altaffilmark{1}}
\altaffiltext{1}{Research School of Astronomy \& Astrophysics, Australian National University, Mount Stromlo Observatory, via Cotter Rd, Weston, ACT 2611, Australia}
\altaffiltext{2}{Massachusetts Institute of Technology, Kavli Institute for Astrophysics and Space Research,
77 Massachusetts Avenue, Cambridge, MA 02139, USA}


\begin{abstract}
\end{abstract}

\keywords{Galaxy: halo, structure --- Individual: Orphan Stream --- Stars: K-giants}

\section{Introduction}

\section{Observations and Data Analysis}

\begin{deluxetable*}{lcccccc}
\tablecolumns{2}
\tabletypesize{\scriptsize}
\tablecaption{Observed Targets\label{tab:observed-targets}}
\tablehead{
	\colhead{Star} &
	\colhead{$\alpha$} &
	\colhead{$\delta$} &
	\colhead{Air mass} &
	\colhead{S/N\tablenotemark{a}} &
	\colhead{$V_{\mbox{helio}}$} &
	\colhead{Comment} \\
 & (J2000) & (J2000) & & (px$^{-1}$) & (km s$^{-1}$) &
}
\startdata
HD 136316 & 15 22 17.2 & $-$53 14 13.9 & 1.118 & 335 & $-38.21 \pm 1.09$ & \\
HD 141531 & 15 49 16.9 & 09 36 42.5 & 1.309 & 280 & $2.63 \pm 1.02$ & \\
HD 142948 & 16 00 01.6 & $-$53 51 04.1 & 1.107 & 271 & $30.26 \pm 0.87$ & \\
HD 41667 & 06 05 03.7 & $-$32 59 36.8 & 1.005 & 272 & $297.79 \pm 1.72$ & \\
HD 44007 & 06 18 48.6 & $-$14 50 44.2 & 1.033 & 239 & $163.41 \pm 1.25$ & \\
HD 47536 & 06 37 47.7 & $-$32 20 20.1 & 1.002 & 257 & $78.87 \pm 1.21$ & \\
HD 76932 & 08 58 44.2 & $-$16 07 54.2 & 1.158 & 289 & $119.18 \pm 1.18$ & \\
HD 84903 & 09 47 19.3 & $-$41 27 04.9 & 1.260 & 294 & $77.65 \pm 1.41$ & \\
HD 59984 & 07 32 05.7 & $-$08 52 56.1 & 1.111 & 402 & $55.74 \pm 0.53$ & \\
HD 60060 & 07 29 59.6 & $-$52 39 04.3 & 1.127 & 414 & $25.60 \pm 1.08$ & \\
HD 60228 & 07 30 30.8 & $-$54 23 58.6 & 1.139 & 338 & $48.58 \pm 1.67$ & \\
OSS TGT 1    & 10 46 50.6 & $-$00 13 17.9 & 1.363 & 48 & $218.15 \pm 1.72$ & \\
OSS TGT 2    & 10 47 17.8 &    00 25 06.9 & 1.995 & 59 & $222.16 \pm 1.28$ & \\
OSS TGT 3    & 10 47 30.3 & $-$00 01 22.6 & 1.156 & 49 & $226.35 \pm 1.27$ & \\
OSS TGT 4    & 10 49 08.3 &    00 01 59.3 & 1.881 & 48 & $227.50 \pm 2.05$ & Poor seeing.\\
OSS TGT 5    & 10 50 33.7 &    00 12 18.3 & 1.295 & 31 & $249.01 \pm 2.58$ & Poor seeing. \\
HR 6141  & 16 30 12.3 & $-$25 06 52.0 & 1.003 & 406 & - & Telluric std.

\enddata
\tablenotetext{a}{S/N measured at 600 nm for each target.}
\end{deluxetable*}


% HDs 59984, 76932, 136316, 84903, 44007, 142948
% OS 2,4,5,6,8 (file name convention) here is OS 1,2,3,4,5

\subsection{Target Selection and Observations}

\subsection{Telluric Absorption Correction}

We observed the spectroscopic white dwarf HR6141 on March 15th, 2011 with the same slit configuration. The airmass during during this observation was 1.003. We assume all detected absorption lines in this spectrum are atmospheric, except for H$\alpha$. Continuum and absorption features in each echelle order were identified by eye and the spectrum was normalised using cubic splines with defined knot spacings. Overlapping echelle orders were stitched together and the normalised telluric template was interpolated onto the wavelength array for our observed stars. Each stellar spectrum was corrected by the normalised template adjusted by the ratio of the air masses, following the Beer-Lambert law:

\begin{equation}
d = \frac{s}{t^{X_{(obs)}/X_{(tel)}}}
\end{equation}

\subsection{Radial Velocities}
We measured radial velocities for every target by cross-correlating each telluric-divided observed spectrum with a synthetic template of $T_{eff} = 4500$ K, $\log{g} = 2.5$, [Fe/H] = -1.5, and [$\alpha$/Fe] = 0.0 over the wavelength region $845$ nm $< \Lambda < 870$ nm. Observed spectra were shifted to rest frame for line measurements. A number of radial velocity and halo standards were observed during this program, and our heliocentric velocities agree excellently with literature values.

\subsection{Line Measurements}

\subsection{Stellar Parameters}

\subsubsection{Effective Temperature}

\subsubsection{Surface Gravity}

\subsection{Uncertainty Analysis}


\section{Abundances}


\section{Discussion}

\section{Results}

\section{Conclusions}







\end{document}
