%!TEX TS-program = pdflatex
\documentclass{emulateapj}

\shorttitle{OSS on MIKE}
\shortauthors{Casey et al}

\begin{document}

\title{High Resolution Spectroscopic Analysis of Orphan Stream Giants}


\author{Andrew R. Casey\altaffilmark{1,2}, Stefan C. Keller\altaffilmark{1}, Frebel, Anna\altaffilmark{2}, Gary Da Costa\altaffilmark{1}, Elizabeth Maunder\altaffilmark{1}}
\altaffiltext{1}{Research School of Astronomy \& Astrophysics, Australian National University, Mount Stromlo Observatory, via Cotter Rd, Weston, ACT 2611, Australia}
\altaffiltext{2}{Massachusetts Institute of Technology, Kavli Institute for Astrophysics and Space Research,
77 Massachusetts Avenue, Cambridge, MA 02139, USA}


\begin{abstract}
\end{abstract}

\keywords{Galaxy: halo, structure --- Individual: Orphan Stream --- Stars: K-giants}

\section{Introduction}

\section{Observations and Data Analysis}

\subsection{Observations}


\begin{deluxetable*}{lcccccc}
\tablecolumns{2}
\tabletypesize{\scriptsize}
\tablecaption{Observed Targets\label{tab:observed-targets}}
\tablehead{
	\colhead{Star} &
	\colhead{$\alpha$} &
	\colhead{$\delta$} &
	\colhead{Air mass} &
	\colhead{Seeing} &
	\colhead{S/N\tablenotemark{a}} &
	\colhead{$V_{\mbox{helio}}$} &
	\colhead{Comment} \\
 & (J2000) & (J2000) & & (") & (px$^{-1}$) & (km s$^{-1}$) &
}
\startdata
HD 136316 & 15 22 17.2 & $-$53 14 13.9 & ? & 1.118 & 335 & $-38.21 \pm 1.09$ & \\
HD 141531 & 15 49 16.9 & 09 36 42.5 & ? & 1.309 & 280 & $2.63 \pm 1.02$ & \\
HD 142948 & 16 00 01.6 & $-$53 51 04.1 & ? & 1.107 & 271 & $30.26 \pm 0.87$ & \\
HD 41667 & 06 05 03.7 & $-$32 59 36.8 & ? & 1.005 & 272 & $297.79 \pm 1.72$ & \\
HD 44007 & 06 18 48.6 & $-$14 50 44.2 & ? & 1.033 & 239 & $163.41 \pm 1.25$ & \\
HD 47536 & 06 37 47.7 & $-$32 20 20.1 & ? & 1.002 & 257 & $78.87 \pm 1.21$ & \\
HD 76932 & 08 58 44.2 & $-$16 07 54.2 & ? & 1.158 & 289 & $119.18 \pm 1.18$ & \\
HD 84903 & 09 47 19.3 & $-$41 27 04.9 & ? & 1.260 & 294 & $77.65 \pm 1.41$ & \\
HD 59984 & 07 32 05.7 & $-$08 52 56.1 & ? & 1.111 & 402 & $55.74 \pm 0.53$ & \\
HD 60060 & 07 29 59.6 & $-$52 39 04.3 & ? & 1.127 & 414 & $25.60 \pm 1.08$ & \\
HD 60228 & 07 30 30.8 & $-$54 23 58.6 & ? & 1.139 & 338 & $48.58 \pm 1.67$ & \\
OSS 1    & 10 46 50.6 & $-$00 13 17.9 & 0.8 & 1.363 & 48 & $218.15 \pm 1.72$ & \\
OSS 2    & 10 47 17.8 &    00 25 06.9 & ? & 1.995 & 59 & $222.16 \pm 1.28$ & \\
OSS 3    & 10 47 30.3 & $-$00 01 22.6 & 0.71 & 1.156 & 49 & $226.35 \pm 1.27$ & \\
OSS 4    & 10 49 08.3 &    00 01 59.3 & ? & 1.881 & 48 & $227.50 \pm 2.05$ & Poor seeing.\\
OSS 5    & 10 50 33.7 &    00 12 18.3 & ? & 1.295 & 31 & $249.01 \pm 2.58$ & Poor seeing. \\
HR 6141  & 16 30 12.3 & $-$25 06 52.0 & ? & 1.003 & 406 & - & Telluric std.
\enddata
\tablenotetext{a}{S/N measured at 600 nm for each target.}
\end{deluxetable*}


% HDs 59984, 76932, 136316, 84903, 44007, 142948
% OS 2,4,5,6,8 (file name convention) here is OS 1,2,3,4,5

\subsection{Target Selection and Observations}


\subsection{Data Reduction}
The data were reduced using the latest version of the MIKE pipeline outlined in \citet{Kelson;2003}. For comparison we also reduced our HD 59984 spectra using standard methods in \textsc{IRAF} and found no noteworthy difference between the reduced products. Individual frames for each object were co-added before analysis. We carefully normalised every echelle order using cubic splines. Overlapping normalised orders on each CCD were stitched together, such that a continuous normalised spectra from each CCD remains. We do not include line measurements that are susceptible to telluric absorption, so no telluric corrections were made.

\subsection{Radial Velocities}
We measured radial velocities for every target by cross-correlating each normalised red  spectrum with a synthetic template. Our synthetic template was generated in \textsc{MOOG} using a \citet{Castelli-Kurucz;2004} model atmospheres with $T_{eff} = 4500$ K, $\log{g} = 2.5$, [Fe/H] = -1.5, [$\alpha$/Fe] = 0.0 and no convective overshoot. The line list of \citet{Kirby;et-al_2008} was employed to generate spectra between $845$ nm $< \Lambda < 870$ nm. Both the blue and red co-added, normalised frames were shifted to rest before equivalent width measurements. A number of radial velocity and halo standards were observed during this program, and our heliocentric velocities agree excellently with literature values (Table \ref{tab:observed-targets}).

\subsection{Line Measurements}




\subsection{Stellar Parameters}


HD59984 5750 K, 1.55 vt, 3.55 logg -0.95,+0.40,
================================================
Fe I EPS 6.52 +/- 0.16 (210 lines)
Slope, Offset, Corr: 0.01, 6.53, 0.05

Fe I REW
Slope, Offset, Corr: 0.0, 6.55, 0.00

Fe I lambda
Slope, Offset, Corr: 3.84e-05, 6.35, 0.17

Fe II EPS 6.53 +/- 0.18 (22 lines)

Fe II REW
Slope, Offset, Corr: 0.07, 6.94, 0.09

Fe II lambda
Slope, Offset, Corr: 1.41e-04, 5.91, 0.36


OSS 1:
161.71091670000001 -0.2216389
SDSS DR 8 closest source is 3.793" away, and 17.47 mag in g
ugriz: 18.823 17.467 16.934 16.712 16.597
No reddenning corrected for, Jester 2005 transformations
UBVRI = 18.06255 17.88487 17.14253 16.68055 16.24855
B -V = 0.74
Estimating a [Fe/H] = -1.9 and using Alonso et al 1999 eq 4 yields 4986 K

With the red arm only:
5110 / 1.20 / 3.10 / -1.14 / 0.40
(with [Fe/H] = -1.14, the estimated temp from photometry is 5014 K)

Fe I Abundance 6.42 +/- 0.19 (106 lines)
S, O, C: 0.00, 6.41, 0.01

Reduced EW S, O, C: 0.10, 6.87, 0.08

Lambda S, O, C: 9.41e-05, 5.89, 0.28

Fe II Abundance: 6.44 +/- 0.17 (8 lines)

Reduced EW S, O, C: -0.09, 6.01, -0.09

Lamba S, O, C: 1.64e-04, 5.53, 0.53

Wiith the red and blue, starting at 5000K, -1.9

5070 / 1.6 / 2.9 / -1.4 / 0.40

Fe I Abundance: 6.15 +/- 0.25 (117 lines)
SOC: 0.01, 6.12, 0.03

Reduced EW SOC: 0.00, 6.12, 0.00
Lambda SOC: 1.41e-04, 5.42, 0.42

Fe II Abundance: 6.14 +/- 0.19 (19 lines)

Reduced EW SOC: -0.04, 5.95, -0.05
Lambda SOC: 1.09e-04, 5.61, 0.41

Followed from this (USING RED AND BLUE data) and excluded outliers.

5040 / 1.4 / 2.8 / -1.2 / 0.40 ("FINAL" files) <-- doesn't quite fit on isochrone, but close-ish
Fe I abundance: 6.24 +/- 0.16 (149 lines)
SOC: 0.02, 6.18, 0.13
REW SOC: 0.03, 6.39, 0.04
Lambda SOC: 0.81e-05, 5.78, 0.38

Fe II abundance: 6.19 +/- 0.16 (15 lines)
REW SOC: 0.03, 6.33, 0.04
Lambda SOC: 1.08e-04, 5.65, 0.46


ALPHA ELEMENTS Mg, Si, Ca, Ti <-> (12, 
==============


Mg I: 6.56 +/- 0.10 (4 lines), Solar 7.60 +/- 0.04
Inspected all four line profiles; they look perfect.
SOC: // // 4.51e-05, 6.29, 0.29
[Mg I/Fe] = 0.16 +/- 0.10

Si I: 6.57 +/- 0.23 (4 lines), Solar 7.51 +/- 0.03
Inspected all four line profiles; they all look great (one is 'reasonable')
SOC: 0.04, 6.36, 0.22 // 0.17, 7.37, 0.28 // 7.75e-05, 6.11, 0.23
[Si I/Fe] = 0.26 +/- 0.23

Ca I: 5.48 +/- 0.14 (14 lines), Solar 6.34 +/- 0.04
Inspected all lines; they all look good.
SOC: // // 1.44e-04, 4.60, 0.70
[Ca I/Fe] = 0.34 +/- 0.14

Ti I: 3.73 +/- 0.15 (21 lines) Solar 4.95 +/- 0.05
Inspected all lines; they all look good.
SOC: // 0.36, 5.49, 0.42 // 2.94e-05, 3.59, 0.08
[Ti I/Fe] = -0.02 +/- 0.15

Ti II: 3.91 +/- 0.10 (25 lines) Solar 4.95 +/- 0.05
SOC: -0.02, 3.91, -0.09 // 0.22, 4.93, 0.43 // -2.70e-05, 4.01, -0.10
[Ti II/Fe] = 0.16 +/- 0.10



OSS 2: JMK = 0.633 (no reddenning correction)
4571.9947676752899 K using Alonso 1999 paper
SDSS (DR8) u g r i z -- searched at 161.8240417 0.4185833, found source 2.05" away
18.063 16.252 15.479, 15.146, 14.950

inspected all profiles + saved only Fe ones that were good in red + blue
================================================
4600 K, 1.85, 0.80, -1.77, 0.40

Fe I: 5.73 +/- 0.14 (139 lines)

Fe I EPS
S, O, C: 0.00, 5.72, 0.01

Fe I REW
S, O, C: -0.03, 5.58, -0.04

Fe I Lambda:
S, O, C: 3.98e-05, 5.51, 0.21

Fe II: 5.72 +/- 0.16 (21 lines)

Fe II EPS: None

Fe II REW S, O, C: 0.03, 5.84, 0.05

Fe II Lambda: 2.1e-05, 5.58, 0.09

=================================
Incorporating isochrones in:
4550 K / 1.85 / 0.8 / -1.80 / 0.40

Fe I Abundance: 5.71 +/- 0.16 (112 lines)
Fe I EPS S, O, C: 0.01, 5.66, 0.10

Fe I REW S, O, C: -0.03, 5.54, -0.04

Fe I Lambda: 8.01e-06, 5.66, 0.03

Fe II Abundance: 5.75 +/- 0.10 (8 lines)

Fe II REW S, O, C: 0.35, 7.52, 0.68

Fe II Lambda = -1.29e-04, 6.48, -0.70

Loaded in all local line list from OSS2/os_4_all_global_auto.moogew

ALPHA ELEMENTS
Ne, Mg, Si, S, Ar, Ca, Ti


Ne I, II -- no lines present in line list

Mg I: 5.90 +/- 0.10 (6 lines), Solar 7.60 +/- 0.04
Inspected all six line profiles; they look perfect.
SOC: 0.05, 5.74, 0.35 // 0.00, 5.93, 0.00 // 3.03e-05, 5.77, 0.15
[Mg I/Fe] = 0.10 +/- 0.10

Si
5565 very small but clean (6 mA)
4102 is great
3095 is a bit blended
Si I: 5.66 +/- 0.26 (3 lines), Solar 7.51 +/- 0.03
Inspected the three lines we have, they don't look fantastic...  3095 could go..
SOC -0.14, 6.14, -0.78 // 0.29, 7.12, 0.92
[Si I/Fe] = -0.05 +/- 0.26

S I, II -- no lines present in line list: they are mainly in the infrared

Ar I, II -- no lines present in line list

Ca I: 4.79 +/- 0.12 (17 lines), Solar 6.34 +/- 0.04
Inspected all the lines I have; they all look reasonable. Outliers could be confidently excluded.
SOC: // 0.48, 7.08, 0.77 // -9.21e-05, 5.30, -0.51
[Ca I/Fe] = 0.25 +/- 0.12

Ti I: 3.00 +/- 0.16 (20 lines), Solar 4.95 +/- 0.05
Inspected all the lines, excluded some dodgey measurements and duplicates, then clipped outliers.
SOC: // -0.04, 2.83, -0.04 // 6.68e-05, 2.69, 0.18
[Ti I/Fe] = -0.15 +/- 0.16

Ti II: 3.19 +/- 0.14 (23 lines), Solar 4.95 +/- 0.05
Inspected all the lines, excluded some dodgey measurements & duplicates then clipped outliers.
SOC: 0.00, 3.23, 0.01 // 0.22, 4.27, 0.28 // -4.71e-05, 3.44, -0.13
[Te II/Fe] = 0.04 +/- 0.14

Na I: 4.21 +/- 0.05 (3 lines), Solar 6.24 +/- 0.04
Inspected all three line profiles; they look perfect.
SOC: -0.06, 4.23, -0.96 // 0.11, 4.71, 0.95
[Na I/Fe] = -0.23 +/- 0.05



OSS 3: 161.8761667 -0.0229444
Not in 2mass (3 arcsecond cone)
SDSS  Closest point is 2.85 arcseconds away at 17.4 mag in g
u g r i z (SDSS DR 8)
18.881 17.400 16.748 16.497 16.316
Using Alonso et al 1999 eq 4 relation, and Jester 2005 transformations:
UBVRI = 18.13946 17.86428 17.00532 16.51173 16.05073
BMV = 0.85895999999999972
Tried inital guess of [Fe/H] as = -1.9 gives 4719 K, giving -1.7 gives 4735 K
using -1.9 guess and starting at 4719 K




Best so far (before excluding outliers)
4800 K, 1.90, 1.70, -1.87, 0.40

after outliers
4810, 1.85, 1.70, -1.87, 0.40

Fe I 5.68 +/- 0.18 (181 lines)
Abundances
S, O, C: 0.00, 5.67, 0.03

Reduced EW
S, O, C: -0.06, 5.38, -0.08

Lambda S, O, C: 8.33e-05, 5.25, 0.34

Fe II 5.67 +/- 0.23 (26 lines)

Reduced EW S, O, C: 0.07, 6.00, 0.09

Lambda S, O, C: 1.44e-04, 4.95, 0.40


===============
with isochrones
4780 / 1.7 / 1.55 / -1.80 / +0.40

Fe I abundance 5.74 +/- 0.13 (112 lines)
S, O, C: 0.00 5.73 0.03
Reduced EW SOC: 0.03 5.86 0.04
Lambda SOC: 3.11e-05, 5.57, 0.13

Fe II abundances 5.69 +/- 0.19 (8 lines)
Reduced EW SOC: -0.14, 5.03, -0.16
Lambda SOC: 1.73e-04, 4.75, 0.49



OSS 4
162.2845417 0.0331389
In SDSS DR8 found closest 16.4 gmag star at 1.113 arcseconds away
u g r i z
17.935 16.440 15.770 15.465 15.299
no redenning, jester transform, assume -1.9 feh and using B-V ~ 0.88
Alonso et al 1999 eq 4
initial start teff = 4693 K

an emp???

4550 /  1.9 / 0.8 / -2.7 / 0.4

Fe I Abundance 4.75 +/- 0.19 (150 lines)
S, O, C: 0.00, 4.76, 0.01

REW S, O, C: 0.04, 4.95, 0.07

Lambda S, O, C: 4.11e-06, 4.74, 0.01

Fe II Abundance 4.78 +/- 0.16 (18 lines)

REW S, O, C: -0.06, 4.51, -0.09

Lambda S, O, C: 9.90e-05, 4.32, 0.33

Using isochrones:

4670 / 1.95 / 1.10 / -2.70 / +0.40
Fe I Abundance 4.84 +/- 0.24 (168)
-0.01, 4.88, -0.03
0.01, 4.93, 0.02
5e-5, 4.61, 0.15

Fe II Abundance 4.87 +/- 0.21 (21)
-0.01, 4.84, -0.01
1.09e-04 4.37 0.27

=============
4580 K / 2.05 / 0.80 / -2.79 / +0.40
( doesn't quite sit on the expected isochrone)

Fe I abundance: 4.75 +/- 0.17 (128 lines)
-0.01 / 4.77 / -0.05
0.03 / 4.88 / 0.04
9.62e-06 / 4.70 / 0.40

Fe II abundance: 4.77 +/- 0.15 (15 lines)
-0.01 / 4.69 / -0.02

6.49e-05 / 4.44 / 0.15

OSS 5
162.6405417 0.2050833
SDSS DR8 nearest target is 1.196" away at 18th magnitude in g
ugriz = 19.628 17.997 17.338 17.065 16.908
No reddenning correction and using Jester 2005:
UBVRI = 18.85619 18.46401 17.59819 17.08062 16.59762
bmv = 0.87

Guessing [Fe/H] = -1.9 yields Teff = 4709 K
Arbitrarily chosing lines only redward of 4400 A

5200 / 2.0 / 2.9 / -0.95 / 0.40

Fe I 6.60 +/- 0.22 (90 lines)
0.00, 6.62, 0.01
0.00, 6.65, 0.00
9.2e-5, 6.12, 0.27

Fe II 6.60 +/- 0.35 (17 lines)
0.12, 7.14, 0.07
9.81e-05, 6.09, 0.20

\subsubsection{Effective Temperature}

\subsubsection{Surface Gravity}

\subsection{Uncertainty Analysis}


\section{Abundances}


\section{Discussion}

\section{Results}

\section{Conclusions}







\end{document}
