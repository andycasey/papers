%!TEX TS-program = pdflatex
\documentclass{emulateapj}

\shorttitle{OSS on MIKE}
\shortauthors{Casey et al}

\begin{document}

\title{High Resolution Spectroscopic Analysis of Orphan Stream Giants}


\author{Andrew R. Casey\altaffilmark{1,2}, Stefan C. Keller\altaffilmark{1}, Frebel, Anna\altaffilmark{2}, Gary Da Costa\altaffilmark{1}, Elizabeth Maunder\altaffilmark{1}}
\altaffiltext{1}{Research School of Astronomy \& Astrophysics, Australian National University, Mount Stromlo Observatory, via Cotter Rd, Weston, ACT 2611, Australia}
\altaffiltext{2}{Massachusetts Institute of Technology, Kavli Institute for Astrophysics and Space Research,
77 Massachusetts Avenue, Cambridge, MA 02139, USA}


\begin{abstract}
\end{abstract}

\keywords{Galaxy: halo, structure --- Individual: Orphan Stream --- Stars: K-giants}

\section{Introduction}

\section{Observations and Data Analysis}

\subsection{Observations}


\begin{deluxetable*}{lcccccc}
\tablecolumns{2}
\tabletypesize{\scriptsize}
\tablecaption{Observed Targets\label{tab:observed-targets}}
\tablehead{
	\colhead{Star} &
	\colhead{$\alpha$} &
	\colhead{$\delta$} &
	\colhead{Air mass} &
	\colhead{Seeing} &
	\colhead{S/N\tablenotemark{a}} &
	\colhead{$V_{\mbox{helio}}$} &
	\colhead{Comment} \\
 & (J2000) & (J2000) & & (") & (px$^{-1}$) & (km s$^{-1}$) &
}
\startdata
HD 136316 & 15 22 17.2 & $-$53 14 13.9 & ? & 1.118 & 335 & $-38.21 \pm 1.09$ & \\
HD 141531 & 15 49 16.9 & 09 36 42.5 & ? & 1.309 & 280 & $2.63 \pm 1.02$ & \\
HD 142948 & 16 00 01.6 & $-$53 51 04.1 & ? & 1.107 & 271 & $30.26 \pm 0.87$ & \\
HD 41667 & 06 05 03.7 & $-$32 59 36.8 & ? & 1.005 & 272 & $297.79 \pm 1.72$ & \\
HD 44007 & 06 18 48.6 & $-$14 50 44.2 & ? & 1.033 & 239 & $163.41 \pm 1.25$ & \\
HD 47536 & 06 37 47.7 & $-$32 20 20.1 & ? & 1.002 & 257 & $78.87 \pm 1.21$ & \\
HD 76932 & 08 58 44.2 & $-$16 07 54.2 & ? & 1.158 & 289 & $119.18 \pm 1.18$ & \\
HD 84903 & 09 47 19.3 & $-$41 27 04.9 & ? & 1.260 & 294 & $77.65 \pm 1.41$ & \\
HD 59984 & 07 32 05.7 & $-$08 52 56.1 & ? & 1.111 & 402 & $55.74 \pm 0.53$ & \\
HD 60060 & 07 29 59.6 & $-$52 39 04.3 & ? & 1.127 & 414 & $25.60 \pm 1.08$ & \\
HD 60228 & 07 30 30.8 & $-$54 23 58.6 & ? & 1.139 & 338 & $48.58 \pm 1.67$ & \\
OSS 1    & 10 46 50.6 & $-$00 13 17.9 & 0.8 & 1.363 & 48 & $218.15 \pm 1.72$ & \\
OSS 2    & 10 47 17.8 &    00 25 06.9 & ? & 1.995 & 59 & $222.16 \pm 1.28$ & \\
OSS 3    & 10 47 30.3 & $-$00 01 22.6 & 0.71 & 1.156 & 49 & $226.35 \pm 1.27$ & \\
OSS 4    & 10 49 08.3 &    00 01 59.3 & ? & 1.881 & 48 & $227.50 \pm 2.05$ & Poor seeing.\\
OSS 5    & 10 50 33.7 &    00 12 18.3 & ? & 1.295 & 31 & $249.01 \pm 2.58$ & Poor seeing. \\
HR 6141  & 16 30 12.3 & $-$25 06 52.0 & ? & 1.003 & 406 & - & Telluric std.

\enddata
\tablenotetext{a}{S/N measured at 600 nm for each target.}
\end{deluxetable*}


% HDs 59984, 76932, 136316, 84903, 44007, 142948
% OS 2,4,5,6,8 (file name convention) here is OS 1,2,3,4,5

\subsection{Target Selection and Observations}


\subsection{Data Reduction}
The data were reduced using the latest version of the MIKE pipeline outlined in \citet{Kelson;2003}. For comparison we also reduced our HD 59984 spectra using standard methods in \textsc{IRAF} and found no noteworthy difference between the reduced products. Individual frames for each object were co-added before analysis. We carefully normalised every echelle order using cubic splines. Overlapping normalised orders on each CCD were stitched together, such that a continuous normalised spectra from each CCD remains. We do not include line measurements that are susceptible to telluric absorption, so no telluric corrections were made.

\subsection{Radial Velocities}
We measured radial velocities for every target by cross-correlating each normalised red  spectrum with a synthetic template. Our synthetic template was generated in \textsc{MOOG}\footnotemark{a}\footnotetext{Sneden's website} using a \citet{Castelli-Kurucz;2004} model atmospheres with $T_{eff} = 4500$ K, $\log{g} = 2.5$, [Fe/H] = -1.5, [$\alpha$/Fe] = 0.0 and no convective overshoot. The line list of \citet{Kirby;et-al_2008} was employed to generate spectra between $845$ nm $< \Lambda < 870$ nm. Both the blue and red co-added, normalised frames were shifted to rest before equivalent width measurements. A number of radial velocity and halo standards were observed during this program, and our heliocentric velocities agree excellently with literature values (Table \ref{tab:observed-targets}).

\subsection{Line Measurements}




\subsection{Stellar Parameters}

HD59984 5750 K, 1.55 vt, 3.55 logg -0.95,+0.40,
================================================
Fe I EPS 6.52 +/- 0.16 (210 lines)
Slope, Offset, Corr: 0.01, 6.53, 0.05

Fe I REW
Slope, Offset, Corr: 0.0, 6.55, 0.00

Fe I lambda
Slope, Offset, Corr: 3.84e-05, 6.35, 0.17

Fe II EPS 6.53 +/- 0.18 (22 lines)

Fe II REW
Slope, Offset, Corr: 0.07, 6.94, 0.09

Fe II lambda
Slope, Offset, Corr: 1.41e-04, 5.91, 0.36


OSS 2: JMK = 0.633
4571.9947676752899 K using Alonso 1999 paper
inspected all profiles + saved only Fe ones that were good in red + blue
================================================
4600 K, 1.85, 0.80, -1.77, 0.40

Fe I: 5.73 +/- 0.14 (139 lines)

Fe I EPS
S, O, C: 0.00, 5.72, 0.01

Fe I REW
S, O, C: -0.03, 5.58, -0.04

Fe I Lambda:
S, O, C: 3.98e-05, 5.51, 0.21

Fe II: 5.72 +/- 0.16 (21 lines)

Fe II EPS: None

Fe II REW S, O, C: 0.03, 5.84, 0.05

Fe II Lambda: 2.1e-05, 5.58, 0.09





\subsubsection{Effective Temperature}

\subsubsection{Surface Gravity}

\subsection{Uncertainty Analysis}


\section{Abundances}


\section{Discussion}

\section{Results}

\section{Conclusions}







\end{document}
